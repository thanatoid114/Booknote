\begin{abstract}
In this thesis, we study two different aspects of many-particle physics.
In the first part, we study the Bose--Einstein condensation of microcavity exciton-polaritons in different artificial lattices.

Bose--Einstein condensation is a quantum phase transition, which allows the system to macroscopically occupy its ground state and develop coherence spontaneously.
Often studied in microcavities, which are optical cavities that trap light at specific wavelengths, exciton-polaritons are a kind of quasiparticle arising from the strong coupling between quantum well excitons and cavity photons.
By periodically aligning cavity pillars in different patterns, one can achieve different artificial lattice structures.

With this setup, we apply the driven-dissipative Gross--Pitaevskii equations to investigate the different consequences of the condensation by changing the pumping schemes and the design of the trapping potentials. Topics include multivalley condensation, phase selection and intermittency of exciton-polariton condensation, flat band condensation, and exciton-polariton topological insulators.

In the second part of this thesis, we focus on the electron-scattering properties of a hybrid Bose--Fermi system.
We consider a system consisting of a spatially separated two-dimensional electron gas layer and an exciton gas layer that interacts via Coulomb forces.
We study the temperature dependence of the system's resistivity with this interlayer electron--exciton interaction and compare the results with the electron--phonon interaction.
\end{abstract}
