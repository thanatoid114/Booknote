\chapter{Summary \& Outlook}\label{Ch7}
We finish this thesis with a general summary about the works we have done and try to give a short discussion about the future possible topics in related fields.

In chapter~\ref{CHONE}, we introduced the basic concepts behind exciton-polaritons~\cite{Deng:2010aa,Kavokin:2007aa,Shelykh:2009aa,Bramati:2013aa}.
We started from a microcavity system in which exciton-polaritons can be found, and then discussed the system's basic Hamiltonian.
We gave a short review of experimental exciton-polariton Bose--Einstein condensation and then discussed the related numerical model to treat such condensation~\cite{Carusotto:2013aa,Keeling:2008js,Wouters:2007aa}.

In Chapter~\ref{CHTWO}, we first considered polaritons in a system in which the cavity photons and excitons were spatially separated in the microcavity.
This special confinement gave us a momentum-dependent coupling between the excitons and cavity photons, which gave rise to unique shape of the dispersion for the ground state.
Further investigations revealed that in the two-dimensional case with TE-TM splitting, a multivalley-dispersion of the ground state was formed~\cite{Sun:2017ab}.
This gives a new technique to achieve valley polarization of exciton-polaritons compared to other methods by applying transition metal dichalcogenide (TMD) monolayers~\cite{Krol:2019aa,Dufferwiel:2017aa,Sun:2017ct}.
% In this chapter, we have considered the formation of exciton-polaritons in a semiconductor microcavity with separate spatially patterned potentials for cavity photons and excitons.
% The separated confinement of cavity photons and excitons allows us a momentum dependent coupling which gives rise to a unique shape of the dispersion in which degenerate ground states appear at non-zero momenta.
% We studied two different limits corresponding to strong and weak energy relaxation.
% In the limit of strong energy relaxation, a simple equilibrium theoretical model predicts spontaneous symmetry breaking in momentum space.
% In the limit of weak energy relaxation, a non-equilibrium model accounting for phonon scattering processes shows the system possess a non-equilibrium condensation at non-zero wave vector.
% Treating exciton polaritons as an open quantum system, we also shown that the correlations between the modes in reciprocal space can occur.
% At last, considering exciton-polaritons in a 2D square lattice, we predict the formation of a multivalley-dispersion.
% Here different valleys exhibit different polarizations, which, in principal, allows us to selectively excite the system by polarized laser and forms a foundation for exciton-polariton valleytronics.
Secondly, we theoretically examined the formation of polariton condensation in a one-dimensional microcavity wire with a periodic, complex-valued potential~\cite{Yoon:2019aa}.
With a generalized Gross--Pitaevskii equation, we showed that condensation can occur in a nontrivial state compared to the known result~\cite{Lai:2007aa}, and under certain conditions, a space-time intermittency phase may appear between two distinct condensate phases~\cite{Hecke:1998aa,Chate:1994aa,PhysRevLett.120.033901}.
% We shown that interacting exciton polaritons loaded into a one-dimensional microcavity wire with a periodic potential and periodic distribution of losses can condense into nontrivial states, where losses are not minimized but maximized.
% Under certain conditions, polaritons can form space-time intermittency phase, which separates two condensate phases with minimal and maximal losses.
% The reconstruction of the condensate wave function takes place by proliferation of dark solitons along the periodic structure.
% The nuclei of the new condensate phase, which are characterized by maximization of losses, are formed with increasing polariton-polariton interaction, and they can be seen as a result of gluing the dark solitons together.

In Chapter~\ref{CHTHREE}, we considered the exciton-polaritons in complex artificial lattices~\cite{Sun:2018aa,Ko:2020aa,Sun:2019ab}.
We studied polariton condensation in a two-dimensional system with a Lieb lattice-shaped potential and investigated the dynamics of the resulting compact localized condensation.
We demonstrated that, unlike the previous works~\cite{Baboux:2016aa,Klembt:2017aa,PhysRevLett.120.097401}, we can use a Laguerre--Gaussian pulse to pump the system near the flat band frequency to excite the compact localized condensates.
This gives us a possible application for Laguerre-Gaussian beam in exciton-polariton physics other than generating the vortex~\cite{Kwon:2019aa}.
We further showed that with incoherent homogeneous background pumping, the coherent compact localized condensates can be maintained longer than the lifetime of the polaritons.
This proposal permits one to construct graphs of compact localized states, similar to the proposals for classical~\cite{Berloff:2017aa} and quantum simulators~\cite{Liew:2018aa,Buluta:2009aa,Georgescu:2014aa}.
% Using an example of realistic two-dimensional exciton-polariton Lieb lattice with distributed losses, we have shown that the (nearly) flat band in this system possesses small but finite dispersion, both in the energy and the lifetime of the states.
% We have demonstrated the possibility to excite compact localized condensates in this nearly FB using resonant Laguerre-Gaussian pulses.
% In spite of small dispersion of the band, the localization and coherence of compact localized condensates remain well defined.
% They exhibit an unusual dynamics, manifested by modulated fast Rabi oscillations.
% The coherent compact localized condensates can be maintained for times much longer than the polariton lifetime in the presence of an incoherent homogeneous background pumping.
Later in this chapter, we considered a polariton system with a honeycomb lattice-shaped potential, compared to the early works~\cite{Bardyn:2015aa,Karzig:2015aa,Klembt:2018aa} we exhibited that it is possible to use a local magnetic field generated by magnetic materials embedded in the microcavity to open a gap in the vicinity of the Dirac point.
Between this gap, we observed one or two nontrivial topological states in the polariton system.
We further showed that the Chern number can be changed by tuning the magnitude of the magnetic field or the intensity of the TE-TM splitting similar to~\cite{Nalitov:2015aa,Bleu:2016aa}.
% We have shown that a local magnetic field due to the presence of a magnetic material can be sufficiently strong to open a gap at the Dirac point and allow for the observation of nontrivial topological states in an exciton-polariton system loaded in a hone ycomb lattice.
% With the change of the intensity of the built-in magnetic field or the TE-TM splitting, the system undergoes a phase transition between two nontrivial states with the Chern numbers $\pm2$ and $\mp1$.
%
% The key advantage of this setup is the size of the system, which can be much smaller than the one requiring a homogeneous external magnetic field.
% This can be highly beneficial for future experiments and applications in devices.
% Furthermore we have studied the Chern numbers and gap sizes as functions of the magnetic flux strength and the peak value of the magnetic field.
% The results show that the design with the use of magnetic material and the regular homogeneous case demonstrate similar behavior.
%
% Furthermore, we have explored the joint effect of the internal magnetic material field with an external magnetic field.
% Depending on the relative direction of the two fields, one can switch between the Chern numbers.
% This switching can be performed ``on the fly", since it is only dependent on a small change of external magnetic field, thereby enabling control over the number and/or direction of the topological edge states.
% By reversing the direction of the external magnetic field, one can also keep the Chern number the same but enlarge the size of the gap significantly, thus increasing the speed of the edge state.
% This allows us to propagate polaritons over longer distances before they decay due to their finite lifetime.

Following these concrete works regarding polariton condensation, there are still some fundamental, intriguing questions in this field.
One interesting topic is the formation and annihilation of vortices in polariton systems~\cite{Keeling:2008js,Lagoudakis:2008aa,Tosi:2011kv,Kwon:2019aa}.
It is known that the decay of the first-order spatial coherence, the recombination of thermally excited vortices, and the temperature dependence of a fractional condensate will be decisive experimental signatures indicating the crossover from BEC to the Berezinskii--Kosterlitz--Thouless phase transition~\cite{Kosterlitz_1972,Kosterlitz_1973,berezinsky1970destruction,berezinsky1972destruction}.
Because of the finite lifetime of polaritons, this crossover in the open driven-dissipative system still needs to be understood~\cite{Roumpos2012}.

Another interesting topic is the compact localized states in polariton systems~\cite{Huber:2010aa,Jacqmin:2014aa,Whittaker:2018ab,Klembt:2017aa}.
With the tight-binding model, the existence of compact localized states gives flat bands, which are a signal of macroscopic degeneracy and diverging density of states for the corresponding Hamiltonian~\cite{Leykam:2018aa,Leykam:2018ab}.
In polariton compact localized condensation, we have an opportunity to study the consequence of interaction in a perturbation-sensitive system~\cite{Kuno_2020,danieli2020manybody}.
Besides, due to the flatness of the dispersion, energy conservation is automatically protected in polariton--polariton interaction, which may be a good environment to enhance the interaction in polariton systems.

As we know, the polariton is a quasiparticle in the strong coupling regime.
Recently, there has been a growing interest to increase the coupling strength between matter and light, which is known as ultrastrong (or deep strong) coupling~\cite{Anappara:2009aa,Gunter:2009aa,Frisk-Kockum:2019aa,Forn-Diaz:2019aa}.
This novel regime reveals more quantum properties in polariton systems and opens a new topic in this field.

As our final work in this thesis, we looked at a different field than in previous chapters. Chapter~\ref{Ch6}  considered the transport of electrons coupled with two-dimensional Bose-condensed dipolar exciton gas by Coulomb interaction~\cite{Kovalev:2013aa,Batyev:2014aa,Boev:2018ab,Sun:2019aa,Villegas:2019aa}.
This Bose-condensed exciton gas can be easily transformed into a polariton BEC system due to the similarity of their interaction terms~\cite{Deng:2010aa,Kavokin:2007aa}.
We calculated resistivity by extending the Bloch--Gr\"{u}neisen approach and provided analytical formulas for both single-bogolon and double-bogolon scattering channels.
Compared to the normal phonon interaction channel, we found that two-bogolon scattering becomes the dominant mechanism in the system at a certain temperature range~\cite{Hwang:2008aa}.
This mechanism provides an opportunity to use bogolon pairs to generate Cooper pairs in the Bardeen--Cooper--Schrieffer phase~\cite{Laussy:2010aa}, which is a subject of future research.
% We have studied the transport of electrons coupled with a two-dimensional Bose-condensed dipolar exciton gas via the Coulomb interaction.
% We calculated the resistivity using and extending the Bloch-Gr\"{u}neisen approach and provided analytical formulas for the single and two-bogolon scattering channels, discovering that two-bogolon scattering can be the dominant mechanism in hybrid systems in certain range of temperatures.
% Furthermore, we suggested an alternative way of electron pairing mediated by a pair of bogolons.
