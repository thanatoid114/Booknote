\chapter{Appendix: Bogolon-mediated electron scattering in graphene}\label{AP:CH6}
% 
\section{Derivation of resistivity via Bloch--Gr\"uneisen approach }
In this Appendix, we derive the low-temperature $T$-dependence of the conductivity of graphene in the hybrid Bose--Fermi system using the Bloch--Gr\"uneisen approach.
%
We start from the Boltzmann equation,
%
\begin{equation}\label{AP6_A1}
e_0\textbf{E}\cdot\frac{\partial f}{\hbar\partial \textbf{p}}=I\{f\},
\end{equation}
%
where $\mathbf{p}$ is the wave vector and we set $p\equiv\abs{\mathbf{p}}$, $\mathbf{E}$ is the perturbing electric field, and $f$ is the distribution function.
%
The scattering integral is given by
%
\begin{eqnarray}
\label{AP6_A2}
I\{f\}&=&-\frac{1}{\hbar}\int\frac{d\textbf{q}d\textbf{p}'}{(2\pi)^2}|M_q|^2\Bigl[N_qf_p(1-f_{p'})\delta(\varepsilon_p-\varepsilon_{p'}+\hbar\omega_q)\delta(\textbf{p}-\textbf{p}'+\textbf{q})\nonumber\\
&{}&{}+(N_q+1)f_p(1-f_{p'})\delta(\varepsilon_p-\varepsilon_{p'}-\hbar\omega_q)\delta(\textbf{p}-\textbf{p}'-\textbf{q})\nonumber\\
&{}&{}+N_qf_{p'}(1-f_{p})\delta(\varepsilon_{p'}-\varepsilon_{p}+\hbar\omega_q)\delta(\textbf{p}'-\textbf{p}+\textbf{q})\nonumber\\
&{}&{}+(N_q+1)f_{p'}(1-f_p)\delta(\varepsilon_{p'}-\varepsilon_{p}-\hbar\omega_q)\delta(\textbf{p}'-\textbf{p}-\textbf{q})\Bigr].
\end{eqnarray}
%
Note that in writing this integral we set the length of the sample $L$ equal to one. In performing a dimensionality analysis, one should include the length squared such that $I\{f\}$ has a dimension of inverse time, as it should.

For small enough electric fields, the electron distribution is not substantially different from the equilibrium Fermi distribution, and thus it can be presented in the form
%
\begin{eqnarray}
f=f^0(\varepsilon_p)-\left(-\frac{\partial f^0}{\partial\varepsilon_p}\right)f^{(1)}_\textbf{p},
\end{eqnarray}
where $f^0$ is the equilibrium Fermi--Dirac distribution and $f^{(1)}_\textbf{p}$ has a dimensionality of energy.
%
Following the steps of the derivation reported in~\cite{Zaitsev:2014aa}, we rewrite:
%
\begin{eqnarray}
\label{AP6_A3}
e_0\textbf{E}\cdot\frac{\partial f}{\hbar\partial \textbf{p}}&=&v_F\frac{e_0\textbf{E}\cdot\textbf{p}}{|\mathbf{p}|}\frac{\partial f^0}{\partial \varepsilon_p}=I\{f^{(1)}_\textbf{p}\};\\\nonumber
I\{f^{(1)}_\textbf{p}\}&=&-\frac{1}{\hbar}\int\frac{d\textbf{q}d\textbf{p}'}{(2\pi)^2}|M_q|^2\frac{1}{\hbar}\frac{\partial N_q}{\partial \omega_q}
\left(f^0(\varepsilon_p)-f^0(\varepsilon_{p'})\right)\\\nonumber
&{}&{}\times\left(f^{(1)}_{\textbf{p}}-f^{(1)}_{\textbf{p}'}\right)
\Bigl[\delta(\varepsilon_{p}-\varepsilon_{p'}-\hbar\omega_q)\delta(\textbf{p}-\textbf{p}'-\textbf{q})\nonumber\\
&{}&{}-\delta(\varepsilon_{p}-\varepsilon_{p'}+\hbar\omega_q)\delta(\textbf{p}-\textbf{p}'+\textbf{q})\Bigr],
\end{eqnarray}
where $$\frac{\partial N_q}{\partial \omega_q}=-\frac{\hbar}{k_BT}N_q(1+N_q).$$
By further integrating over the electron wave vector $\textbf{p}'$, we find
%
\begin{eqnarray}
\label{AP6_A4}
v_F\frac{e_0\textbf{E}\cdot\textbf{p}}{p}\frac{\partial f^0}{\partial \varepsilon_p}
=&-&\frac{1}{\hbar}\int\frac{d\textbf{q}}{(2\pi)^2}|M_q|^2\frac{1}{\hbar}\frac{\partial N_q}{\partial \omega_q}
\left(f^0(\varepsilon_p)-f^0(\varepsilon_{p}-\hbar\omega_q)\right)\\ \nonumber
&\times&\left(f^{(1)}_{\textbf{p}}-f^{(1)}_{\textbf{p}-\textbf{q}}\right)
\delta(\varepsilon_{\textbf{p}}-\varepsilon_{\textbf{p}-\textbf{q}}-\hbar\omega_\textbf{q})\\
\nonumber
&+&\frac{1}{\hbar}\int\frac{d\textbf{q}}{(2\pi)^2}|M_q|^2\frac{1}{\hbar}\frac{\partial N_q}{\partial \omega_q}
\left(f^0(\varepsilon_p)-f^0(\varepsilon_{p}+\hbar\omega_q)\right)\\ \nonumber
&\times&\left(f^{(1)}_{\textbf{p}}-f^{(1)}_{\textbf{p}+\textbf{q}}\right)
\delta(\varepsilon_{\textbf{p}}-\varepsilon_{\textbf{p}+\textbf{q}}+\hbar\omega_\textbf{q}).
\end{eqnarray}
%
Let the electric field be directed along the $x$-axis; then, we can use the correction function in the form
\begin{eqnarray}
f^{(1)}_\textbf{p}= v_F\frac{e_0E_xp_x}{k_F}\tau(\varepsilon_p),
\end{eqnarray}
%
where $p_F$ is the Fermi wave vector and $\tau(\varepsilon_p)$ is the relaxation time.  
We now have
%
\begin{eqnarray}\label{AP6_A5}
\frac{p_x}{p}\frac{\partial f^0}{\partial \varepsilon_p}
=&-&\frac{1}{\hbar}\int\frac{d\textbf{q}}{(2\pi)^2}|M_q|^2\frac{1}{\hbar}\frac{\partial N_q}{\partial \omega_q}
\left[f^0(\varepsilon_p)-f^0(\varepsilon_{p}-\hbar\omega_q)\right] \nonumber \\ 
&\times&\left[\frac{p_x}{k_F}\tau(\varepsilon_p)-\frac{p_x-q_x}{k_F}\tau(\varepsilon_p-\hbar\omega_q)\right]
\delta(\varepsilon_{\textbf{p}}-\varepsilon_{\textbf{p}-\textbf{q}}-\hbar\omega_\textbf{q})\nonumber\\
&+&\frac{1}{\hbar}\int\frac{d\textbf{q}}{(2\pi)^2}|M_q|^2\frac{1}{\hbar}\frac{\partial N_q}{\partial \omega_q}
\left[f^0(\varepsilon_p)-f^0(\varepsilon_{p}+\hbar\omega_q)\right] \nonumber \\ 
&\times&\left[\frac{p_x}{k_F}\tau(\varepsilon_p)-\frac{p_x+q_x}{k_F}\tau(\varepsilon_p+\hbar\omega_q)\right]
\delta(\varepsilon_{\textbf{p}}-\varepsilon_{\textbf{p}+\textbf{q}}+\hbar\omega_\textbf{q}).
\end{eqnarray}
%
Assuming that the relaxation time is constant~\cite{Ziman:2001aa} at $\tau=\tau_0$, we find
%
\begin{eqnarray}
\label{AP6_ABoltzmann1}
\frac{k_F}{p}p_x\frac{\partial f^0}{\partial \varepsilon_p}
=&-&\frac{\tau_0}{\hbar}\int \frac{d\textbf{q}}{(2\pi)^2}q_x|M_q|^2\frac{1}{\hbar}\frac{\partial N_q}{\partial \omega_q}\\ \nonumber
&\times&\left[f^0(\varepsilon_p)-f^0(\varepsilon_{p}-\hbar\omega_q)\right]
\delta(\varepsilon_{\textbf{p}}-\varepsilon_{\textbf{p}-\textbf{q}}-\hbar\omega_\textbf{q})\\
\nonumber
&+&\frac{\tau_0}{\hbar}\int \frac{d\textbf{q}}{(2\pi)^2}q_x|M_q|^2\frac{1}{\hbar}\frac{\partial N_q}{\partial \omega_q}\\ \nonumber
&\times&\left[f^0(\varepsilon_p)-f^0(\varepsilon_{p}+\hbar\omega_q)\right]
\delta(\varepsilon_{\textbf{p}}-\varepsilon_{\textbf{p}+\textbf{q}}+\hbar\omega_\textbf{q}).
\end{eqnarray}
%

Let us denote the angle between vectors $\textbf{p}$ and $\textbf{q}$ as $\varphi$, and the angle between vectors $\textbf{p}$ and $\textbf{E}$ as $\beta$. This gives $q_x=q\cos(\varphi+\beta)$ and $p_x=p\cos\beta$. Integrating over $\phi$, we obtain
%
\begin{eqnarray}
\label{AP6_AEq19}
\int_0^{2\pi} &d\phi&\cos (\phi+\beta)\delta(a-\sqrt{b^2\pm c^2\cos\phi})\\\nonumber
&=&\pm\frac{4|a|(b^2-a^2)\Theta[c^4-(b^2-a^2)^2])}{c^2\sqrt{c^4-(b^2-a^2)^2}}\cos\beta,
\end{eqnarray}
%
where $\Theta[x]$ is the Heaviside step function, $a=\hbar(v_Fp-sq)$, $b^2=\hbar^2v_F^2(p^2+q^2)$, and $c^2=2\hbar^2 v_F^2pq$. 
To derive Eq.~\eqref{AP6_AEq19}, we denote a new variable $x=\cos\phi$. This implies $d\phi=\mp dx[1-x^2]^{-1/2}$, where the $-$($+$) case is for $0\leq\phi <\pi$ ($\pi\leq\phi<2\pi$).
%
After integrating over the angle $\phi$, we can integrate Eq.~\eqref{AP6_ABoltzmann1} over $\xi_p=\varepsilon_p-\mu$ using
%
\begin{eqnarray}
\int\limits_{-\infty}^{\infty}d\xi_p\left(f^0(\varepsilon_p)-f^0(\varepsilon_{p}\pm\hbar\omega_q)\right)&=&\mp\hbar\omega_q,\nonumber \\
\int\limits_{-\infty}^{\infty}d\xi_p\frac{\partial f^0}{\partial \varepsilon_p}&=&-1,
\end{eqnarray}
%
and putting all electron wave vectors to be $p=k_F$.

Resistivity is inversely proportional to scattering time by
%
\begin{eqnarray}
\label{AP6_Arho1}
\rho\propto\frac{1}{\tau_0}&=&\frac{\hbar\xi_I^2}{8\pi^2k_FM}\frac{1}{kT}\int_0^\infty dq q^4e^{-2ql}q\nonumber\\
&\times&(\Gamma_--\Gamma_+)_{k_F}N_q(1+N_q),
\end{eqnarray}
%
where we introduce $\xi_I= e_0^2d\sqrt{n_c}/2\epsilon$ and
%
\begin{eqnarray}
\label{AP6_Agamma}
\Gamma_{\pm}=\frac{4|a_{\pm}|(a^2_{\pm}-b^2)\Theta[c^4-(a^2_{\pm}-b^2)^2]}{c^2\sqrt{c^4-(a^2_\pm-b^2)^2}}.
\end{eqnarray}
%
The subscript $k_F$ in the expression $(\Gamma_--\Gamma_+)_{k_F}$ in Eq.~\eqref{AP6_Arho1} means that all the electron wave vectors $p$ are to be substituted by the Fermi value $k_F$.
%

We now introduce a new dimensionless variable,
%
\begin{eqnarray}
\label{AP6_Dim}
u=\frac{\hbar sq}{k_BT},
\end{eqnarray}
%
in Eq.~\eqref{AP6_Arho1} and obtain
%
\begin{eqnarray}
\label{AP6_Arho2}
\frac{1}{\tau_0}&=&\frac{\xi_I^2}{8\pi^2k_FMs}\left(\frac{k_BT}{\hbar s}\right)^4\int_0^\infty du \frac{u^4e^{(1-2\tilde{l})u}}{(e^u-1)^2}\nonumber\\
&\times&(\Gamma_--\Gamma_+)_{k_F},
\end{eqnarray}
%
where
%
\begin{eqnarray}
\Tilde{l}=\frac{lk_BT}{\hbar s}\sim\frac{k_BT}{10\mbox{ meV}}.
\end{eqnarray}
%
We set $s=10^5$ m/s and $l= 5.0\times 10^{-8}$ m/s. 
%
Note that room temperature is $k_BT_R\sim 26$ meV, so that for far lower temperatures we have $\Tilde{l}\ll 1$. Hence, we can replace
%
\begin{eqnarray}
\label{AP6_approx1}
e^{(1-2\tilde{l})u}\rightarrow e^u.
\end{eqnarray}
%

Let us now look at the Heaviside theta function argument in Eq.~\eqref{AP6_Agamma}. It can be simplified to
%
\begin{eqnarray}
-(v_F^2-s^2)q^2\pm 4k_Fsv_Fq+4k_F^2v_F^2.
\end{eqnarray}
%
This expression is positive for
%
\begin{eqnarray}
0\leq q<\frac{2k_Fv_F}{v_F\mp s}\approx 2k_F,
\end{eqnarray}
%
or
%
\begin{eqnarray}
\label{AP6_Alambda}
0\leq u<\frac{T_{BG}}{T}\equiv\Lambda,
\end{eqnarray}
%
where $T_{BG}=2\hbar sk_F/k_B$ is the Bloch--Gr\"{u}neisen temperature for bogolons.

Let us consider the case in which temperature $T\ll T_{BG}$, which specifically means $\Lambda >10$. This inequality gives us the precise form of what we mean by \textit{low temperature}: $k_BT/E_F<10^{-2}$. For typical $E_F\sim 10^{-1}$ eV, this gives $T_{BG}=183$ K and  $T<18$ K (for a particular distance between the layers of $l>50$ nm). 
%
It should be mentioned that the condition $\tilde{l}\ll 1$ is not a requirement. However, the general case does not allow for the analytical extraction of temperature out of the integral, since we come up with the term $\sim\exp[lk_BTu/(\hbar s)]$ under the integration.

For large $u$ (or $q$), the factors $\Gamma_\pm$ in Eq.~\eqref{AP6_Arho2} approach constant values. In the meantime, the term $u^4\exp({-u})$ rapidly goes to zero for $u>10$.
%
Therefore, we can remove the theta function in Eq.~\eqref{AP6_Agamma} and only incur a small (imaginary) error. Doing this and also using $v_F^2-s^2\sim v_F^2$, Eq.~\eqref{AP6_Agamma} now becomes
%
\begin{eqnarray}
\label{AP6_Agamma2}
\Gamma_\pm=\frac{2|v_Fk_F\pm sq|(2v_Fsk_F-v_F^2q)}{\hbar v_F^3k_F q\sqrt{\pm 4k_Fsv_Fq+4k_F^2v_F^2-v_F^2q^2}},
\end{eqnarray}
%
where the expression in the numerator can be rewritten as
\begin{eqnarray}
2v_Fsk_F-v_F^2q&=&2v_Fsk_F-v_F^2\frac{k_BT}{\hbar s}u\nonumber\\
&=&\frac{v_Fk_BT}{\hbar s}(s\Lambda-v_Fu).
\end{eqnarray}
%
Here, $\Lambda$ does depend on $T$, as was defined in Eq.~\eqref{AP6_Alambda}. For $T\ll T_{BG}$ though, due to the factor $\exp({-u})$, we can simply replace $\Lambda\sim 10$ (or greater) without significantly affecting the result. 
%
We introduce the ratio of velocities $\alpha=s/v_F$ and for simplicity, we choose $\Lambda$ such that $\alpha\Lambda=1$ as long as $\Lambda>10$, to then get
%
\begin{eqnarray}
2v_Fsk_F-v_F^2q
&=&\frac{v_Fk_BT}{\alpha\hbar }(1-u).
\end{eqnarray}
%
We find
%
\begin{eqnarray}
\frac{1}{\tau_0}&=&\frac{10v_F\xi_I^2}{8\pi^2k_FMs}\left(\frac{k_BT}{\hbar s}\right)^5\nonumber\\
&\times&\int_0^\infty du \frac{u^4e^u(1-u)}{(e^u-1)^2}(\gamma_--\gamma_+)_{k_F},
\end{eqnarray}
%
where
%
\begin{eqnarray}
\label{AP6_AEqMath}
\gamma_\pm=\frac{2|v_Fk_F\pm sq|}{\hbar v_F^3k_F q\sqrt{\pm 4k_Fsv_Fq+4k_F^2v_F^2-v_F^2q^2}}.
\end{eqnarray}
%
The term in the absolute value can be rewritten as
%
\begin{eqnarray}
|v_Fk_F\pm sq|=\frac{k_BT}{\hbar}\left|\frac{\Lambda}{2\alpha}\pm u\right|.
\end{eqnarray}
%

Finally, the term under the square root in Eq.~\eqref{AP6_AEqMath} can be written as 
%
\begin{eqnarray}
\nonumber
\pm 4k_Fsv_Fq&+&4k_F^2v_F^2-v_F^2q^2\nonumber\\
&\sim&-v_F^2(q-2k_F)(q+2k_F)\nonumber\\
&=&-\frac{v_F^2k_B^2T^2}{\hbar^2s^2}(u-\Lambda)(u+\Lambda).
\end{eqnarray}
%

Summing up, we find
%
\begin{eqnarray}
\frac{1}{\tau_0}&=&\frac{I_0\xi_I^2k_F^2}{4\pi^2\hbar \alpha^4v_F^2M}\left(\frac{k_BT}{E_F}\right)^4\nonumber\\
&=&\frac{5I_0e_0^2}{8\pi^2\hbar^3 v_F}\frac{M}{n_cE_F^2}(k_BT)^4,
\end{eqnarray}
%
where $I_0$ is a dimensionless integral, which can be found numerically by
%
\begin{eqnarray}
I_0&=&\int_0^\infty du\frac{u^3(1-u)e^u}{(e^u-1)^2\sqrt{100-u^2}}\nonumber\\
&\times&\left(\left|\frac{\Lambda}{2\alpha}-u\right|-\left|\frac{\Lambda}{2\alpha}+u\right|\right)
\approx 26.2.
\end{eqnarray}
%
This gives
\begin{eqnarray}
\frac{1}{\tau_0}=(1.4\times 10^{16}\mbox{s}^{-1})\left(\frac{k_BT}{E_F}\right)^4.
\end{eqnarray}
%
In terms of the resistivity,
%
\begin{eqnarray}
\label{AP6_Arho3}
\rho=\frac{\pi\hbar^2}{e_0^2E_F}\frac{1}{\tau_0}=(1.0\times 10^6\;\Omega)\left(\frac{k_BT}{E_F}\right)^4.
\end{eqnarray}
%
Thus, we conclude that at low temperatures $T\ll T_{BG}$, the resistivity is $\propto T^4$.


