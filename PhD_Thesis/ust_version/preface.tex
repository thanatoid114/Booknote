\begin{preface}
\begin{center}
 和子由渑池怀旧\\
    \vspace{2mm}\hspace{40mm}苏轼\\
 \vspace{2mm}人生到处知何似,应似飞鸿踏雪泥。\\
 \vspace{2mm}泥上偶然留指爪,鸿飞那复计东西。\\
 \vspace{4mm}老僧已死成新塔,坏壁无由见旧题。\\
 \vspace{2mm}往日崎岖还忆否,路长人困蹇驴嘶。
\end{center}
    \vspace{5mm}
\begin{center}
    A reply to Zi You's ``In Memory of Mian Chi Days"\\
    \vspace{2mm}\hspace{40mm} Su Shi\\
\end{center}
     How to describe the diverse human life? It is like a swan alighting on the muddy snow.\\
     By chance the snow is imprinted by his feet. After he flys away, who knows where he went to?\\ 
\vspace{2mm}\\
     The old monks died and became new pagodas, the crumbling walls could no longer keep our old verses.\\
     Do you still remember the hard days in the past? Endless roads, exhausted men, crippled braying donkeys.
\end{preface}
