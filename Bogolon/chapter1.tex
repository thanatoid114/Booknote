\chapter{Basic properties}
In what follows, we have $\hbar = k_B=1$.

\section{Bogoliubov coefficient}\label{sec:BovCoef}
Bogoliubov coefficient is defined in~\cite{giorgini:98di}

\begin{eqnarray}\label{eq:bov_coe}
&&u^2_{\mathbf{p}}=1+v^2_{\mathbf{p}}=\frac{1}{2}\left(1+\left[1+\left(\frac{Ms^2}{\omega_{\mathbf{p}}}\right)^2\right]^{1/2}\right),
\\
&&u_{\mathbf{p}}v_{\mathbf{p}}=-\frac{Ms^2}{2\omega_{\mathbf{p}}},
\end{eqnarray}
where the $M$ is the effective mass of the condensed particle; $s$ is the sound velocity of bogolons and $s= \sqrt{\frac{\kappa n_c}{M}}$; the $n_c$ is the density of condensed particle; $\kappa$ is the interaction strength; and the dispersion of the bogolons are\footnote{
Usually, if we consider the exciton condensation, $\kappa$ is exciton-exciton interaction, for indirect exciton, the result is $\kappa = \frac{e_0^2 d}{\epsilon}$.
}
%
\begin{equation}\label{eq:bov_disp}
  \omega_\bp = s\bp \sqrt{1+\bp^2 \xi_h^2}.
\end{equation}
The healing length is defined as $\xi_h=\frac{1}{2Ms}$.

\section{Green's function}\label{sec:Green}
Bogolons are bosons.
The Green's function is defined like phonons,
First, introducing the following operators
\begin{equation}\label{eq:Aq}
  A_\bq = u_\bq b_\bp + v_\bq b^\dagger_{-\bq}
\end{equation}
unlike the phonons where $u_\bq =v_\bq=1$, we do \textbf{not} have $A_\bq^\dagger = A_{-\bq}$.
These corresponding operators are
\begin{eqnarray}
  A_{-\bq} &=& u_\bq b_{-\bq} + v_\bq b^\dagger_\bq \label{eq:Aq1} \\
  A_{\bq}^\dagger &=& u_\bq b_\bq^\dagger + v_\bq b_{-\bq} \label{eq:Aq2}
\end{eqnarray}
where we see from \eqref{eq:bov_coe}, $u_\bp$ and $v_\bp$ are only magnitude depended.

From here, we have two definitions of the Green's function
\begin{equation}
  \cf(\bq,\tau) = -\langle T A_\bq(\tau) A_{-\bq} \rangle \label{eq:N_Green_tau}
\end{equation}
The corresponding Green's function in Matsubara frequency is
\begin{equation}
  \cd(\bq,i\omega_n) =u_\bq v_\bq \left[ \frac{1}{i\omega_n-\omega_\bq} - \frac{1}{i\omega_n + \omega_\bq} \right]= \frac{2 u_\bq v_\bq \omega_\bq}{(i\omega_n)^2 -\omega_\bq^2}  \label{eq:N_Green}
\end{equation}
However, if we consider this
\begin{equation}
  \cg(\bq,\tau) = - \langle T A_\bq(\tau) A_\bq^\dagger \rangle \label{eq:F_Green_tau}
\end{equation}
The corresponding result is
\begin{equation}
  \cg(\bq,i\omega_n) = \frac{u_\bq^2}{i\omega_n - \omega_\bq} - \frac{v_\bq^2}{i\omega_n + \omega_\bq} \label{eq:F_Green}
\end{equation}
With another notation, we found
\begin{equation}
  \cg'(\bq,\tau) = -\langle T A^\dagger_\bq(\tau)A_\bq \rangle \label{eq:F_Green2_tau}
\end{equation}
the result is
\begin{equation}
  \cg'(\bp,i\omega_n) = \frac{v^2_\bq}{i\omega_n-\omega_\bq} - \frac{u_\bq^2}{i\omega_n + \omega_\bq} \label{eq:F_Green2}
\end{equation}
We can make a table for the result of the Green's function in Matsubara frequency

\begin{center}
{\Large
\begin{fullwidth}
\begin{tabular}{||c|c|c|c|c|c||}
    \hline
    $\int d\tau e^{i\omega_n \tau}$ & $-\langle T A_\bq(\tau) \cdot$ & $ -\langle T A_\bq^\dagger(\tau) \cdot$ & $-\langle T A_{-\bq}(\tau) \cdot$ & $ -\langle T A_{-\bq}^\dagger(\tau) \cdot$ & $\times$ \\
    \hline
    ~ & $0$ &  \textcolor{NavyBlue}{$\frac{v_\bq^2}{i\omega_n -\omega_\bq} - \frac{u_\bq^2}{i\omega_n+\omega_\bq}$} & $\frac{u_\bq v_\bq}{i\omega_n-\omega_\bq} - \frac{u_\bq v_\bq}{i\omega_n + \omega_\bq}$ & \textcolor{NavyBlue}{$\frac{v_\bq^2}{i\omega_n -\omega_\bq} - \frac{u^2_\bq}{i\omega_n+\omega_\bq}$} & $A_\bq\rangle$ \\
    \hline
    ~ & \textcolor{OliveGreen}{$\frac{u_\bq^2}{i\omega_n-\omega_\bq} - \frac{v_\bq^2}{i\omega_n+\omega_\bq}$} & $0$ & \textcolor{OliveGreen}{$\frac{u_\bq^2}{i\omega_n-\omega_\bq} - \frac{v_\bq^2}{i\omega_n+\omega_\bq}$} & $\frac{u_\bq v_\bq}{i\omega_n - \omega_\bq} - \frac{u_\bq v_\bq}{i\omega_n + \omega_\bq}$ & $A_\bq^\dagger \rangle$ \\
    \hline
    ~ & $\frac{u_\bq v_\bq}{i\omega_n-\omega_\bq} - \frac{u_\bq v_\bq}{i\omega_n + \omega_\bq}$ & \textcolor{NavyBlue}{$\frac{v_\bq^2}{i\omega_n -\omega_\bq} - \frac{u_\bq^2}{i\omega_n+\omega_\bq}$} & $0$ & \textcolor{NavyBlue}{$\frac{v_\bq^2}{i\omega_n -\omega_\bq} - \frac{u^2_\bq}{i\omega_n+\omega_\bq}$} & $A_{-\bq}\rangle$ \\
    \hline
    ~ & \textcolor{OliveGreen}{$\frac{u_\bq^2}{i\omega_n-\omega_\bq} - \frac{v_\bq^2}{i\omega_n+\omega_\bq}$} & $\frac{u_\bq v_\bq}{i\omega_n-\omega_\bq} - \frac{u_\bq v_\bq}{i\omega_n + \omega_\bq}$ & \textcolor{OliveGreen}{$\frac{u_\bq^2}{i\omega_n-\omega_\bq} - \frac{v_\bq^2}{i\omega_n+\omega_\bq}$} & $0$ & $A^\dagger_{-\bq} \rangle$\\
    \hline
\end{tabular}
\end{fullwidth}
}
\end{center}

By looking the \eqref{eq:F_Green} and \eqref{eq:F_Green2}, we realized that this two Green's function are identical
\begin{equation}
  \cg(\bq,i\omega_n) = \cg'(\bq,-i\omega_n) \label{eq:F_Green_realition}.
\end{equation}
Using the \eqref{eq:bov_coe}, with some calculation we can write down the Green's function in a matrix form~\cite{christensen:15qp}
\begin{equation}
\hat{\cg}=
  \begin{pmatrix}
    \cg & \cf \\
    \cf & \cg'
  \end{pmatrix}
\end{equation}

Another method to calculate the Green's function is based on the the discussion in~\cite{giorgini:98di,kovalev:13lo}
The result is given as the retarded Green's function
\begin{equation}\label{Green_ret_vad}
\hat{\cg}_{ret} =
  \begin{pmatrix}
    \frac{E+(q^2/2M) + \kappa n_c}{E^2-\omega_\bq^2 + i\delta} & \frac{-\kappa n_c}{E^2-\omega_\bq^2 + i\delta} \\
    \frac{-\kappa n_c}{E^2-\omega_\bq^2+i\delta} & \frac{-E+(q^2/2M)+ \kappa n_c}{E^2 - \omega_\bq^2+i\delta}
  \end{pmatrix}
\end{equation}
This result is given by the following approximation:
\begin{eqnarray}
    && \frac{u_\bq^2}{i\omega_n + \omega_\bq} - \frac{v_\bq^2}{i\omega_n + \omega}\nonumber \\
    &=& \frac{(u_\bq^2-v_\bq^2) i\omega_n + (v_\bq^2 + u_\bp^2) \omega_\bq}{(i\omega_n)^2 -(\omega_\bq)^2} \nonumber \\
    &=& \frac{E+i\delta + (v_\bq^2 + u_\bq^2)\omega_\bq}{E^2 + 2i\delta E - \delta^2 - \omega_\bq^2} \nonumber \\
    &\approx& \frac{E + (u_\bq^2 + v_\bq^2)\omega_\bq }{E^2 -\omega^2 +i\delta \mathrm{sign}(E)}
\end{eqnarray}
where we first apply the analytic continuous: $i\omega_n \to E + i \delta$ and assme $\delta E \to 0$.
Further, for the rest part we consider
\begin{eqnarray}
    u_\bq^2 &=& \frac{1}{2} \left( 1 + \frac{Ms^2}{\omega_\bq} \sqrt{1 + \frac{\omega_\bp^2}{M^2 s^4} } \right) \nonumber \\
    &\approx& \frac{1}{2} \left( 1 + \frac{Ms^2}{\omega_\bq} \left( 1 + \frac{\omega_\bq^2}{2 M^2 s^4} \right) \right) \nonumber \\
    &=& \frac{1}{2} \left( 1 + \frac{Ms^2}{\omega_\bq} + \frac{\omega_\bq}{2 M s^2}  \right) \\
    v_\bq^2 &=& \frac{1}{2} \left( \frac{Ms^2}{\omega_\bq} + \frac{\omega_\bq}{2 M s^2}  -1 \right)
\end{eqnarray}
we can get the final result of the first element in \eqref{Green_ret_vad}.
