chapter{Optical Properties of Solids}

\section{Wannier Excitons}\label{s9.2}
%
\subsection{The Model}
Exciton states play an extremely important role in the understand of interband transitions in semiconductors.
The word exciton is used to signify the modification of the \textit{absorption rate of phonons due the the Coulomb interaction between the electron and the valenec band hole}.

\begin{figure}[ht]
    \centering
    \begin{tikzpicture}
        \node at (-1.5,4.5) {a};
        \draw [->] (0,0) -- (0,4);
        \draw [->] (-1,0.2) -- (1,0.2);
        \node [below] at (1,0.2) {$k$};
        \draw (-1.5,0) .. controls (-0.25,2) and (0.25,2) .. (1.5,0);
        \draw (-1.5,4) .. controls (-0.25,2) and (0.25,2) .. (1.5,4);
        \draw (0,1.525) -- (1,1.525);
        \node [right] at (1,1.55) {$E=0$};
        \draw [->] (-1,0.9) -- (-1,3.1);
        \node at (3,4.5) {b};
        \draw [snake it] (3,0)--(3,2);
        \draw [middlearrow=latex] (3,2) -- (2,3);
        \draw [middlearrow=latex] (3,2) -- (4,3);
        \node [above] at (2,3) {$e^-$};
        \node [above] at (4,3) {$h^+$};
    \end{tikzpicture}
    \caption{Optical transition in a semiconductor between occupied valence band and empty conduction band for a direct transition. (a) Conventional band picture; (b) Wannier picture where the photon makes an electron-hole.}%
    \label{fig:9.6}
\end{figure}
%
The easiest case is shown in Fig.\ref{fig:9.6}(a), the interband transition is direct.
The valance band states are all filled and the conduction band states are all empty.
The vertical arrow show a possible interband transition which can occur when a photon is absorbed in the solid.
The valence state is shown as nondegenerate (except for spin) at $\bk=0$, although that is seldom the case; usually the band has an orbital degeneracy and is anisotropic.

\textit{The point of view in Fig.\ref{fig:9.6}(a) is a single-particle picture of the transition process}.
The transition rate for the absorption of photons is given by golden rule
\begin{eqnarray}
    A(\omega) &=& \frac{2\pi}{\hbar V} \sum_{\bk \bk'} \abs{ \bra{c,\bk'} \hat{\varepsilon} \cdot \bp \ket{v,\bk} }^2 \delta \left[\varepsilon_v(\bk) + \hbar \omega - \varepsilon_c(\bk') \right]      \label{9.73} \\
    \varepsilon_v(\bk) &=& - \frac{k^2}{2m_v},  ~ ~ ~ ~ \varepsilon_c(\bk) = E_g + \frac{k^2}{2m_c}  \label{9.74}
\end{eqnarray}
The energy zero is chosen to be the top of the valence band.
The matrix element is between the one-electron initial and final states.
The wave function are taken to be Bloch function $u_{\bk}(\br) \otimes\exp(i\bk \br)$.
For discussion reason assume the cell-periodic parts as independent of wave vector $\bk$, and assumed that the valance band has $p$ symmetry and conduction band has $s$ symmetry.
Then $u_v(\br)$ is a periodic orbital with angular momentum $l=1$, while $u_c(\br)$ is $l=0$.
With these approximations the optical matrix element is a constant except for wave vector conservation
\begin{eqnarray}
    \ket{v,\bk} &=& u_v(\br) \frac{e^{i\bk\br}}{\sqrt{V}} \label{9.75} \\
    \ket{c,\bk'} &=& u_v(\br) \frac{e^{i\bk'\br}}{\sqrt{V}} \label{9.76} \\
    \bra{c,\bk'} \hat{\varepsilon} \bp \ket{v,\bk} &\equiv& \hat{\varepsilon} \bp_{cv} \delta_{\bk\bk'} \label{9.77} \\
    \hat{\varepsilon}\bp &=& \frac{1}{V} \int_{cell} d^3 r u_c^* \hat{\varepsilon} \bp u_v  \label{9.78}
\end{eqnarray}
neglect the wave vector $\bq$ for photons since it is in the optical frequencies.
\begin{equation}
    A(\omega) = \abs{\hat{\varepsilon} \bp_{cv} }^2 \frac{(2\mu)^{3/2}}{2\pi} \sqrt{\hbar \omega - E_g} \Theta(\hbar \omega -E_g)   \label{9.79}
\end{equation}
with $\mu^{-1} = m_c^{-1} + m_v^{-1}$.
Equation \eqref{9.79} predicts that the absorption rate begins at the energy gap of semiconductor and rise as the square root of factor of optical frequency.
But this is not observed, in fact, the one-particle theory is totally inadequate.

Wannier observed that the interband transition in semiconductors was really a \textbf{two-particle process}, as in Fig.\ref{fig:9.6}(b).
In this new picture, the electron and hole are particles with charges of opposite sign, so that there is a Coulomb attraction $- \frac{^2}{\varepsilon_0 r}$ between them, where $\varepsilon_0$ is the static dielectric function.
The dielectric function is assumed to be a constant which is independent of the frequency.\footnote{This is a poor approximation, since most semiconductors are polar and dielectric function has significant dispersion at frequencies near the optical phonon frequencies, see \ref{s6.3}}
The attractive Coulomb interaction between the electron and hole can cause hydrogenic bound states between them.

The optical absorption rate for this process was calculated by Elliott. The final state of the system is described by a two-particle Sch{\"o}dinger equation
\begin{eqnarray}
    &&\Psi(\br_e,\br_h) = u_c(\br_e) u_v(\br_h) \Phi(\br_e,\br_h)     \label{9.82}    \\
    &&0 =\left[ - \frac{\hbar^2\nabla_e^2}{2m_c} - \frac{\hbar^2\nabla_h^2}{2m_v}  - \frac{e^2}{\varepsilon_0 \abs{\br_e-\br_h}} -E  \right] \Phi(\br_e,\br_h)   \label{9.83}
\end{eqnarray}
$\Phi(\br_e,\br_h)$ can be factored into relative $\br = \br_e-\br_h$ and center of mass coordinate $M=m_c+m_v$ in standard fashion
\begin{eqnarray}
    \mathbf{R} &=& \frac{m_c \br_e+m_v \br_h}{M}        \label{9.84} \\
    \Phi(\br_e,\br_h) &=& \frac{e^{i\mathbf{PR}}}{\sqrt{V}} \phi(\br)    \label{9.85} \\
    0 &=& \left( - \frac{\hbar^2}{2\mu} \nabla^2 - \frac{e^2}{\varepsilon_0 \br} -\varepsilon_r \right) \phi(\br)       \label{9.86} \\
    E &=& E_g + \varepsilon_r + \frac{P^2}{2M}  \label{9.87}
\end{eqnarray}
The center of mass motion is plane-wave-like, with a wave vector $\mathbf{P}$ which in optical experiments is equal to the photon wave vector.
This is usually small and set $\mathbf{P}=0$.
For relative energy $\varepsilon_r<0$, the two particles form bound hydrogenic states with energy $\varepsilon_r = \varepsilon_n = - \frac{E_R}{n^2} $.  For relative energy greater than zero, the form scattering states $\phi_\bk(\br)$.
Elliott showed that the optical transition rate depends on the relative wave function $br=0$.
Instead of \eqref{9.79}, the transition rate is
\begin{equation}
    A(\omega) = \frac{2\pi}{\hbar} \abs{\hat{\varepsilon} \bp_{cv}}^2 \sum_j \abs{\phi_j(0)}^2 \delta(\hbar \omega - E_g -\varepsilon_j)  \label{9.88}
\end{equation}
The summation $j$ run over the bound and continuum states.

The relative motions of electron and hole are in s-wave hydrogenic state, either bound or unbound, because of the angular momentum selection rule.
The one-unit change in $l$, in the photon absorption, is take by the change of band symmetry, and the relative motion is not permitted any additional angular momentum.
For $s$ states, the bound states have an amplitude given by the principal quantum number $n$ and the Bohr radius $a_0$,
\begin{equation}
    \phi_n(0) = \frac{1}{\sqrt{\pi a_0^3 n^3}} \label{9.89}
\end{equation}
For continuum states, with energy $\varepsilon_k = \frac{k^2}{2\mu} $, the relative wave function at the origin is
\begin{equation}
    \psi_{k,l=0}(0) = \frac{2\pi \eta}{V \left[2- e^{-2\pi \eta} \right]}
\end{equation}
where $\eta^{-1} = k a_0$.
Then \eqref{9.89} predicts that the absorption is a constant in frequency at the energy gap $E_g$, and does not rise with a square root dependence.
The absorption function now have a few sharp, distinct exciton lines at low frequency correspond to $1s$, $2s$, etc.
These absorption bands are very strong, and all the light is attenuated before trasversing the sample.
At higher frequencies, the $ns$ stats are closer in frequency and are broadened and merges with the continuum absorption which starts at $\hbar \omega = E_g$.

\subsection{Solution by Green's Functions}

