\chapter{Nonzero Temperatures}

\section{Introduction}\label{s3.1}
At nonzero temperature, whether electron, phonon, or spin, is interacting with a bath of other particles which have an average energy.
The exact state of all these other particles is not known, since they are fluctuating between different configurations.
All that is know is the temperature, which is related to the mean energy.

\marginnote{
Important knowledge, the expansion
\begin{eqnarray*}
  &&\frac{1}{e^x-1}\\
  &=&\frac{1}{1+x+\frac{x^2}{2!}+\frac{x^3}{3!}+\dots -1}\\
  &=&\frac{1}{x+\frac{x^2}{2!}+\frac{x^3}{3!}+\dots}\\
  &=&\frac{1}{x(1+\frac{x}{2!}+\frac{x^2}{3!}+\dots)}\\
  &\approx&\frac{1}{x(1+\frac{x}{2})}\\
  &\approx&\frac{1}{x}(1-\frac{x}{2}) = \frac{1}{x}-\frac{1}{2}
\end{eqnarray*}
}

When defining the Green's function, one must average over all possible configurations of the system. A possible Green's function for electron is
\begin{eqnarray}
  \frac{\Tr[e^{-\beta H} C_{\bp \sigma}(t) C^\dagger_{\bp \sigma}(t')]}{Tr(e^{-\beta H})}  \label{3.1} \\
  C_{\bp \sigma}(t) = e^{itH}C_{\bp\sigma}e^{-itH}  \label{3.2}
\end{eqnarray}
where "$\Tr$" denotes trace and is summation over some complete set of states.

The Matsubara method treats time as a complex temperature, the object is treat $t$ and $\beta$ as the real and imaginary parts of a complex variable, which will require only one $S$-matrix expansion.

The expansion of the series for bosons and fermions give
\begin{eqnarray}
  \eta_F(\xi_\bp) &=& \frac{1}{e^{\beta \xi_\bp}+1} = \frac{1}{2} + \frac{1}{\beta} \sum_{n=-\infty}^\infty \frac{1}{(2n+1)i\pi/\beta -\xi_\bp} \label{3.5} \\
  \eta_B(\omega_\bq) &=& \frac{1}{e^{\beta \omega_\bq}-1} = -\frac{1}{2} + \frac{1}{\beta} \sum_{n=-\infty}^\infty \frac{1}{2ni\pi/\beta -\omega_\bq} \label{3.6}
\end{eqnarray}
\textit{These series can be derived from a theorem which states that any meromorphic function may be expanded as a summation over its poles and residues at those poles}. \footnote{\textcolor{red}{Need to be more clear about this.}}
It is convenient to define the frequencies at the pole
\begin{equation}
  p_n = (2n+1)\pi/\beta,~ ~ ~ ~ \omega_n = 2n\pi/\beta \label{3.7}
\end{equation}

In Matsubara method, time becomes a complex quantity which is usually called $\tau$, where $\tau = it$.
Green's functions are function of $\tau$ with domain
\begin{equation}
  \label{3.10}
  -\beta \leq \tau \leq \beta
\end{equation}
Fourier transform theory states that if a function $f(\tau)$ is defined over such a range, then its Fourier expansion is
\begin{equation}
  \label{3.11}
  f(\tau) = \frac{1}{2}a_0 + \sum_{n=1}^\infty \left[a_n \cos(\frac{n\pi\tau}{\beta}) + b_n \sin(\frac{n\pi\tau}{\beta})\right]
\end{equation}
where
\begin{eqnarray}
  a_n &=& \frac{1}{\beta} \int_{-\beta}^\beta d\tau f(\tau) \cos(\frac{n\pi\tau}{\beta}) \label{3.12} \\
  b_n &=& \frac{1}{\beta} \int_{-\beta}^\beta d\tau f(\tau) \sin(\frac{n\pi\tau}{\beta}) \label{3.13}
\end{eqnarray}
or \textbf{equivently}. we have
\begin{equation}
  f(i\omega_n) = \frac{1}{2} \beta(a_n + ib_n ) \label{3.14}
\end{equation}
with
\begin{eqnarray}
  f(\tau)&=&\frac{1}{\beta} \sum_{n=-\infty}^\infty e^{-in\pi\tau/\beta} f(i\omega_n) \label{3.15} \\
  f(i\omega_n)&=& \frac{1}{2} \int_{-\beta}^\beta d\tau f(\tau) e^{in\pi\tau/\beta}   \label{3.16}
\end{eqnarray}

\subsection{Boson}
There is still a further simplification can be achieved.
For boson Green's functions have the additional property that
\begin{equation}
  \label{3.17}
  \mathrm{boson:}~ f(\tau) = f(\tau+ \beta),~ ~ -\beta<\tau<0
\end{equation}
Further we have
\begin{eqnarray}
    f(i\omega_n)&=&\int_0^\beta d\tau e^{i\omega_n\tau} f(\tau) \nonumber \\
    f(\tau) &=& \frac{1}{\beta} \sum_n e^{-i\omega_n\tau} f(i\omega_n) \label{3.20} \\
    \omega_n &=& 2n\pi k_BT\nonumber
\end{eqnarray}

\subsection{Fermion}
Similarly, the fermion Green's function will have the property that
\begin{equation}
  \label{3.21}
  \mathrm{fermions:}~ f(\tau) = - f(\tau+ \beta),~ ~ -\beta<\tau<0
\end{equation}
and the result give
\begin{eqnarray}
  f(i\omega_n)&=&\int_0^\beta d\tau e^{i\omega_n\tau}f(\tau) \nonumber \\
  f(\tau)&=&\frac{1}{\beta}\sum_n e^{-i\omega_n\tau}f(i\omega_n) \label(3.23) \\
  \omega_n &=&(2n+1)\pi k_B T \nonumber
\end{eqnarray}

\section{Matsubara Green's functions}\label{s3.2}
The electron Green's function is defined as
\begin{eqnarray}
  \cg(\bp,\tau-\tau') &=& - \langle T_\tau C_{\bp\sigma}(\tau)C^\dagger_{\bp\sigma}(\tau') \rangle  \label{3.24} \\
  \cg(\bp,\tau-\tau')&=& -\Tr \big[e^{-\beta(H-\mu N -\Omega)} T_\tau e^{\tau(H-\mu N)} C_{\bp\sigma}e^{-(\tau-\tau')(H-\mu N)} \nonumber \\
  &\times& C^\dagger_{\bp\sigma} e^{-\tau'(H-\mu N)} \big] \label{3.25} \\
  e^{-\beta \Omega} &=& \Tr(e^{-\beta(H-\mu N)}) \label{3.26}
\end{eqnarray}
The definition is equivalent between \eqref{3.24} and \eqref{3.25}, and it is the thermodynamic average, which is the trace over the complete set of states.
The $\mu$ is the chemical potential and $N$ is the particle number operator.
A grand canonical ensemble is used, where the number of particles is variable.
This many-particle system can be successfully used for one particle in an empty band.
In this case, the analytical continuation si taken as $i\omega_n \to E+ \mu + i\delta$, and the chemical potential will vanish from all expressions.
One is not bothered by the fact that $\beta \mu \ll 0$ in one-particle systems at nonzero temperatures.

In a many-electron system, the chemical potential is retained in the formalism.
The analytical continuation is $i\omega_n \to E+ i\delta$ and energy is measured from the chemical potential.

With some derivation\footnote{P.112~P.114}, we have the Green's functions
\begin{eqnarray}
  \cg(\bp,\tau)&=&-\langle T_\tau C_{\bp\sigma}(\tau) C^\dagger_{\bp\sigma}(0) \rangle \label{3.33}\\
    &=& -\Tr \left[ e^{-\beta (K-\Omega)} T_\tau (e^{\tau K} C_{\bp\sigma} e^{-\tau K} C^\dagger_{\bp\sigma})\right] \label{3.34} \\
  \cg(\bp,i\omega_n) &=& \int_0^\beta d\tau e^{i\omega_n \tau} \cg(\bp,\tau)  \label{3.39} \\
  \cg(\bp,\tau) &=& \frac{1}{\beta} \sum_n e^{-i\omega_n \tau} \cg(\bp,i\omega_n) \label{3.40}
\end{eqnarray}

For noninteracting Green's function, the $\tau$ evolution of the operators is\footnote{Derived from the Baker-Hausdorff theorem
\begin{equation*}
  e^ACe^{-A} = C+ [A,C] + \frac{1}{2!}[A,[A.C]]+\dots
\end{equation*}}
\begin{eqnarray}
  C_{\bp\sigma}(\tau) &=& e^{\tau K_0} C_{\bp\sigma} e^{-\tau K_0} = e^{-\xi_\bp \tau} C_{\bp \sigma } \label{3.44} \\
  C^\dagger_{\bp\sigma}(\tau) &=& e^{\tau K_0} C^\dagger_{\bp\sigma} e^{-\tau K_0} = e^{\xi_\bp \tau} C_{\bp \sigma }^\dagger \label{3.45}
\end{eqnarray}
Then the Green's function is
\begin{equation}
  \cg^0(\bp,\tau)= -e^{-\xi_\bp \tau}[\Theta(\tau)-\eta_F(\xi_\bp)] \label{3.49}
\end{equation}
and
\begin{equation}
  \cg^0(\bp,i\omega_n) = \frac{1}{i\omega_n -\xi_\bp}   \label{3.55}
\end{equation}

The phonon and photon Green's functions are defined in the same fashion,
\begin{eqnarray}
    \cd (\bq,\tau-\tau') &=& - \langle T_\tau \mathbf{A}(\bq,\tau) \mathbf{A}(-\bq,\tau') \rangle \label{3.56} \\
    \mathbf{A}(\bq,\tau) &=& e^{\tau H}(a_\bq+ a^\dagger_{-\bq})e^{-\tau H} \label{3.57}
\end{eqnarray}
and with the relation in \eqref{3.17}, we have
\begin{equation}
  \cd (\bq,\tau) = \cd (\bq, \tau+\beta)~ ~ ~ ~ -\beta<\tau<0 \label{3.63}
\end{equation}

For noninteracting system the Green's function of \textbf{phonons} is
\begin{equation}
  \cd^0(\bq,i\omega_n) = -\frac{2\omega_\bq}{\omega_n^2 + \omega_\bq} \label{3.76}
\end{equation}
Notice that it is almost identical to the zero-temperature case \eqref{2.73}.

The \textbf{photon} Green's function is also identical to its zero-temperature result, except for complex frequencies.
The  fundamental definition is
\begin{eqnarray}
  \cd_{\mu\nu}(\bk,\tau) &=& -\sum_\lambda \langle T_\tau \mathbf{A}_\mu(\bk,\lambda,\tau) \mathbf{A}_\nu(-\bk,\lambda,0)\rangle \label{3.77}\\
  \mathbf{A}_\mu(\bk,\lambda,0) &=& \xi_\mu(\bk,\lambda) (\frac{2\pi}{\omega_\bk})(a_{\bk \lambda}+ a^\dagger_{\bk \lambda}) \label{3.78}\\
  \cd^0_{\mu\nu}(\bk,i\omega_n)&=&-\frac{4\pi(\delta_{\mu\nu}-k_\mu k_\nu/k^2)}{\omega_n^2+\omega_\bk^2} \label{3.79}
\end{eqnarray}

\section{Retarded and advance Green's functions}\label{s3.3}
The retarded and advanced Green's functions were introduced in \ref{s2.9}.
All measurable quantities, such as conductivities or susceptibilities, are actually retarded correlation functions.
The Green's function by Matsubara function can be easily convert to retarded and advanced Green's function.

The retarded Green's functions may be defined for both zero and nonzero temperature.
The retarded Green's function for an electron in state $\bp$ is
\begin{eqnarray}
  G_{ret}(\bp,t-t') &=& G_t(\bp,t-t') - G^<(\bp,t-t')  \\
  &=&-i\Theta(t-t')\langle [C_{\bp\sigma}(t)C^\dagger_{\bp\sigma}(t') + C^\dagger_{\bp\sigma}(t')C_{\bp\sigma}(t)] \rangle \nonumber \\
  &=&-i\Theta(t-t')\Tr \{ e^{-\beta (K-\Omega)} [C_{\bp\sigma}(t) C^\dagger_{\bp\sigma}(t') + C^\dagger_{\bp\sigma}(t')C_{\bp\sigma}(t)]\} \nonumber \label{3.82} \\
  K&=&H-\mu N, ~ ~ ~ ~ C_{\bp\sigma}(t) = e^{iKt}C_{\bp\sigma}e^{-iKt} \label{3.83}
\end{eqnarray}
The Green's function operates only for $t>t'$, which makes it \textbf{causal}.
In the limit that times becomes equal, the anticommutator becomes unity.
The plus sign in the middle of the two terms is an important feature for retarded Green's functions of fermion operators.

For phonons, the retarded Green's function is
\begin{equation}
  D_{ret}(\bq,t-t') = -i\Theta(t-t')\langle A(\bq,t)A(-\bq,t') - A(-\bq,t')A(\bq,t)\rangle \label{3.85}
\end{equation}
The sign in the middle is minus, which corresponding the bosons' commutation relations.

Retarded Green's functions are needed for many types of operators with products of electron or boson operators.
\begin{eqnarray}
  U&=& \sum_{ij}M_{ij}C^\dagger_iC_j \label{3.86}\\
  V&=& \sum_{ijk}M_{ijk}C^\dagger_i C_j C_k \label{3.87}
\end{eqnarray}
The operator $U$ is bilinear in operator $C_i$, this can be regarded as having boson properties.
However, operator $V$ can be regarded as fermion operators.
\begin{eqnarray}
  \bar{U}_{ret}(t-t') &=& -i\Theta(t-t')\langle [U(t)U^\dagger(t') - U^\dagger(t')U(t)] \rangle \label{3.88} \\
  \bar{V}_{ret}(t-t') &=& -i\Theta(t-t')\langle [V(t)V^\dagger(t') + V^\dagger(t')V(t)] \rangle \label{3.89}
\end{eqnarray}
All these retarded function have the Fourier transforms defined by the usual convention:
\begin{equation}
  G_{ret}(\bp,E) = \int_{-\infty}^\infty dt e^{iE(t-t')} G_{ret}(\bp,t-t') \label{3.90}
\end{equation}

THe advanced Green's function for each is defined
\begin{eqnarray}
  G_{adv}(\bp,t-t')&=&i\Theta(t-t')\langle [C_{\bp\sigma}(t)C^\dagger_{\bp\sigma}(t') \nonumber \\
  &+& C^\dagger_{\bp\sigma}(t')C_{\bp\sigma}(t)] \rangle \label{3.93} \\
  D_{adv}(\bq,t-t') &=& i\Theta(t-t')\langle A(\bq,t)A(-\bq,t') \nonumber \\
  &-& A(-\bq,t')A(\bq,t)\rangle \label{3.94}
\end{eqnarray}
The only two differences are the sign change in front and that the time domain is now $t'>t$, which is just opposite of that for retarded function.

The advanced functions of energy is defined as usual Fourier transform, and turn out to be \textit{complex conjugate} of the corresponding retarded function.
\begin{equation}
  U_{ret}(\omega) = U_{adv}^\dagger(\omega) \label{3.98}
\end{equation}

The Matsubara function can be changed to a retarded one with just this alteration:
\begin{equation}
  i\omega_n \to \omega + i\delta~ ~ ~ ~ \mathcal{U}(i\omega_n) = U_{ret}(\omega)  \label{3.110}
\end{equation}
This step is called an analytic continuation.
The advanced Green's function can be changed by $i\omega_n \to \omega - i\delta$, since the advanced Green's function is the complex conjugate of the retarded one.

Another quantity is \textbf{spectral function}, it is imaginary part of any retarded function multiplied by 2:
\begin{equation}
  A(\bp,\omega) = -2\Im [G_{ret}(\bp,\omega)]  \label{3.115}
\end{equation}
An expression of this form is called \textbf{Lehmann representation}:
\begin{eqnarray}
  U_{ret}(\omega) &=& \int_{-\infty}^\infty \frac{d\omega'}{2\pi} \frac{A(\omega')}{\omega-\omega'+i\delta} \nonumber \\
  \mathcal{U}(i\omega_n) &=& \int_{-\infty}^\infty \frac{d\omega'}{2\pi} \frac{A(\omega')}{i\omega_n -\omega'} \label{3.118}
\end{eqnarray}

For fermions, the spectral density function is positive, $A(\bp,\omega)>0$.
This positiveness is an important feature, since $A(\bp,\omega)$ is interpreted as a probability function.
\begin{equation}
    1 = \int \frac{d\omega}{2\pi} A(\bp,\omega) \label{3.121}
\end{equation}
For bosons spectral function do not have this property, however, they are always positive for $\omega>0$ and negative for $\omega<0$.

For noninteracting electron the Green's function is
\begin{equation}
  G_{ret}^0 (\bp,E)= \frac{1}{E-\xi_\bp + i\delta} \label{3.129}
\end{equation}
It has one sign $\delta>0$, even in may electron system with a Fermi surface.
The retarded functions do not have $\delta_\bp$ changing sign at the Fermi surface, which makes them easier to use than the zero-temperature Green's function.

The spectral function for the noninteracting Green's function is \footnote{\begin{equation*}
\frac{1}{E-\xi_\bp + i\delta} = \frac{1}{E-\xi_\bp} - 2\pi i \delta(E-\xi_\bp)
\end{equation*}}
\begin{equation}
    A^0(\bp, E) = 2\pi \delta(E-\xi_\bp) \label{3.130}
\end{equation}
When $A(\bp,E)$ is computed for interacting systems, there is a broad of the delta function.
This means there is a band of $E$ values for each $\bp$,.
When the electron scatters, it has a nonzero mean free path, and there is some uncertainty in its momentum or energy or both.
So $bp$ and $E$ are treated as separate variables and both are summed over when evaluating physical quantities.

Another quantity to evaluate, for an interacting electron system, is the number of electrons in a momentum state $\bp$, which is
\begin{equation}
  n_\bp = \int \frac{dE}{2\pi} \eta_F(E) A(\bp,E) \label{3.135}
\end{equation}

For phonons, the average number of phonons in a state $\bq$ is
\begin{equation}
  2N_\bq +1  = \langle A^\dagger_\bq A_\bq \rangle = \int \frac{d\omega}{2\pi} \eta_B(\omega)A(\bq,\omega) \label{3.136}
\end{equation}
The noninteracting phonon spectral function is
\begin{equation}
  A^0(\bq,\omega) = 2\pi [\delta(\omega-\omega_\bq) - \delta(\omega+ \omega_\bq)] \label{3.137}
\end{equation}

In later sections the Green's function in Matsubara form have a Dyson form
\begin{eqnarray}
    \cg(\bp,ip_n) &=& \frac{1}{ip_n -\xi_\bp -\Sigma(\bp,ip_n)}     \label{3.138} \\
    \cd (\bq,i\omega_n) &=& \frac{-2\omega_\bq}{\omega_n^2 + \omega_\bq^2 +2\omega_\bq P(\bq,i\omega_n)} \label{3.139}
\end{eqnarray}
Define the retarded self-energies according to \eqref{3.110}
\begin{equation}
  ip_n \to E + i\delta ~ ~ ~ ~ \Sigma(\bp,ip_n) \to \Sigma_{ret}(\bp,E) = \Re \Sigma_{ret} + i \Im \Sigma_{ret}    \label{3.140}
\end{equation}
Consider the retarded Green's function will have a Dyson equation
\begin{equation}
  G_{ret}(\bp,E) = \frac{1}{E+i\delta-\xi_\bp - \Sigma_{ret}(\bp,E)} \label{3.141}
\end{equation}
as derived from \eqref{3.110}.
The spectral function for the electron is rewritten in terms of the retarded self-energy,
\begin{equation}
  A(\bp,E) = \frac{-2 \Im \Sigma_{ret}(\bp,E)}{[E-\xi_\bp - \Re \Sigma_{ret}(\bp,E)]^2 + [\Im\Sigma_{ret}(\bp,E)]^2}  \label{3.142}
\end{equation}

Simple examples to distinguish Matsubara, retarded, and advanced Green's function.
There are some simple functions which have the correct analytical properties.
Considering a self-energy operator has the following functional form,
\begin{equation}
  \Sigma(\bp,Z) = C \ln [f(\bp) -Z] \label{3.143}
\end{equation}
where $Z$ is a complex variable representing the frequency. Take $C$ as a constant and $f(\bp)$ as some function of momentum.
The Matsubara self-energy is evaluated at the points
\begin{equation}
  \Sigma(\bp,ip_n) = C \ln [f(\bp)-ip_n] \label{3.144}
\end{equation}
The analytic continuation $ip_n \to E  \pm i\delta$ to the real axis has the following values.
For retarded function \footnote{
To get this relation
\begin{eqnarray*}
    \Sigma_{ret} &=& C \ln [f(\bp)-E-i\delta]\\
    &=& C \int \frac{dE}{f(\bp)-E-i\delta} + \mathbf{C}
\end{eqnarray*}
}, using $ip_n \to E + i \delta$
\begin{equation}
  \Sigma_{ret}(\bp,E) = C \ln \abs{f(\bp) -E} - i\pi C \Theta[E-f(\bp)] \label{3.145}
\end{equation}
and for advanced
\begin{equation}
  \Sigma_{adv(\bp,E)} = C \ln \abs{f(\bp)-E} + i\pi C \Theta[E-f(\bp)]  \label{3.146}
\end{equation}
These two self-energies differ in the region $E>f(\bp)$, because their imaginary parts have the opposite sign. This difference agrees with the general theorem that
\begin{equation}
  G^*_{ret}(\bp,E) = G_{adv}(\bp,E) \label{3.147}
\end{equation}
and also implies that
\begin{equation}
  \Sigma_{ret}^*(\bp,E) = \Sigma_{adv}(\bp,E) \label{3.148}
\end{equation}
There is a branch cut on the real axis for $E>f(\bp)$. This branch cut just expresses the fact that $\ln(f-Z)$ is not a continuous function of $Z$ across the real axis.

Another example which has the similar analytical properties is
\begin{equation}
  \Sigma(\bp,Z) = C[f(\bp) -Z]^{\frac{1}{2}} \label{3.149}
\end{equation}
This function also has a branch cut for $E>f(\bp)$, with $\Im\Sigma<0$ and $\Im\Sigma>0$.
In fact, a branch cut is a necessary feature whenever $\Im \Sigma \neq 0$, which gives
\begin{equation}
  \Sigma(\bp,E+i\delta) \neq \Sigma(\bp,E-i\delta)  \label{3.150}
\end{equation}
One should be aware that when self-energy functions are evaluated, they are often given by logarithmic or square root function.

When a branch cut occurs and $\Im\Sigma \neq 0$, then the spectral function is given by \eqref{3.142}.
In other regions where there is no branch cut, then take the limit of $\Im\to 0$ and obtain
\begin{equation}
  \lim_{\Im\Sigma=0}A(\bp,E)=2\pi \delta(E-\xi_\bp-\Re\Sigma_{ret}(\bp,E))   \label{3.151}
\end{equation}
Here the spectral function is a delta function, but the real part of the self-energy may be nonzero, so that it affects the spectral function.
Denote by $E(\bp)$ the solution to the equation
\begin{equation}
  E(\bp) -\mu = \xi_\bp + \Re\Sigma(\bp,E(\bp)-\mu) \label{3.152}
\end{equation}
Assume that there is a problem in which \eqref{3.152} is satisfied when $\Im\Sigma=0$.
Then with the properties of delta function, the spectral function is written as
\begin{eqnarray}
  A(\bp,E)&=&2\pi Z(\bp) \delta(E-E(\bp)+\mu) \label{3.154} \\
  Z(\bp)&=&\frac{1}{\abs{1- \frac{\partial}{\partial E} \Sigma_{ret}(\bp,E)}_{E=E(\bp)-\mu}}  \label{3.155}
\end{eqnarray}
The factor $Z(\bp)$ is called \textbf{renormalization factor}.
Because of \eqref{3.121} and $A(\bp,E)>0$, we have $Z(\bp)<1$.
The strength of the delta function peak is always less than or equal to unity.

Equation \eqref{3.152} may be used to define the effective mass. Assume that the noninteracting states are free particles
\begin{equation}
  \xi_\bp = \frac{p^2}{2m} -\mu = \varepsilon_\bp -\mu  \label{3.156}
\end{equation}
Furthermore, assume at low momentum that $E(\bp)$ in \eqref{3.152} varies quadratically wit momentum,
\begin{equation}
  E(\bp) = E_0 + \frac{p^2}{2m^*} +\mathcal{O}(p^4)  \label{3.157}
\end{equation}
The proportionality constant is the inverse effective mass $m^*$
\begin{equation}
  \frac{m}{m^*} = \frac{\partial E(\bp)}{\partial \varepsilon_\bp}  \label{3.158}
\end{equation}
from the definition of \eqref{3.152}, we get
\begin{equation}
  \frac{\partial E(\bp)}{\partial \varepsilon_\bp} = \lim_{\varepsilon_\bp \to 0} \left(1 + \frac{\partial}{\partial \varepsilon_\bp} \Re\Sigma_{ret}(\bp,E) + \left[ \frac{\partial}{\partial E}\Re\Sigma_{ret}(\bp,E) \right] \frac{\partial E(\bp) }{\partial \varepsilon_\bp} \right)  \label{3.159}
\end{equation}
With \eqref{3.158}, the final term is arrived
\begin{equation}
  \frac{m}{m^*} = \lim_{\varepsilon_\bp \to 0} \left[ \frac{1 + (\partial_{\varepsilon_\bp}) \Re \Sigma_{ret}(\bp,E_0-\mu)}{1 - (\partial_{E_0}) \Re \Sigma_{ret}(\bp,E_0-\mu)}\right]  \label{3.160}
\end{equation}

\section{Frequency Summations}
When using the Matsubara Green's functions, one must evaluate frequency summations over combinations of unperturbed Green's functions.
The technique for evaluating these summations is discussed for both cases of unperturbed functions and also Green's functions with self-energies.
There is a table of results for combinations which often occur.
\begin{eqnarray}
  -\frac{1}{\beta} \sum_m \cd^0(\bq,i\omega_m)\cg^0(\bp,ip_n+i\omega_m) &=& \frac{N_\bq + \eta_F(\xi_\bp)}{ip_n-\xi_\bp+\omega_\bq} \nonumber \\
  &+& \frac{N_\bq + 1 - \eta_F(\xi_\bp)}{ip_n -\xi_\bp-\omega_\bq} \label{3.216} \\
  \frac{1}{\beta} \sum_n \cg^0(\bp,ip_n) \cg^0(\bk,ip_n+i\omega_m) &=& \frac{\eta_F(\xi_\bp)-\eta_F(\xi_\bk)}{i\omega_m + \xi_\bp -\xi_\bk} \label{3.217} \\
  -\frac{1}{\beta} \sum_n \cg^0(\bp,ip_n) \cg^0(\bk,i\omega_m-ip_n) &=& \frac{1-\eta_F(\xi_\bp) - \eta_F(\xi_\bk)}{i\omega_m -\xi_\bp -\xi_\bk} \label{3.218}\\
  \frac{1}{\beta} \sum_n \cg^0(\bp,ip_n) &=& \eta_F(\xi_\bp)\label{3.219}
\end{eqnarray}

First consider the summation over a boson series, considering \eqref{3.216},
\begin{equation}
  S = \frac{1}{\beta} \sum_m \frac{2\omega_\bq}{\omega_m^2+\omega_\bq^2} \frac{1}{ip_n+i\omega_m -\xi_\bp} \label{3.220}
\end{equation}
Denote this equation as
\begin{equation}
  S = -\frac{1}{\beta} \sum_m f(i\omega_m) \label{3.221}
\end{equation}
where $f(i\omega_m)$ is the product of Green's function in \eqref{3.220}. This summation si evaluated by a contour integration. The integral has the form,
\begin{equation}
  I = \lim_{R\to \infty} \oint \frac{dz}{2\pi i} f(z) \eta_B(z)  \label{3.222}
\end{equation}
The function $\eta_B(z)$ is chosen to generate poles at the points $i\omega_m$ for all even integer $m$.
