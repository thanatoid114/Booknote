\chapter{Introductory Material}

\section{Harmonic oscillators and phonons}{se:1.1}
First quantization in physics refers to the property of particles that certain operators do not commute
\begin{eqnarray}
    \left[ x, p_x \right] &=& i\hbar  \label{1.1} \\
    E &\to& i\hbar \frac{\partial}{\partial t}  \label{1.2}
\end{eqnarray}
For the second quantization, the basic idea is that forces are caused by the exchange of particle, and the number of particles is quantized, for example photons for electromagnetic force.

In solids the quantized vibrational modes of the atom are quantized because of the first quantization \eqref{1.1}.
This vibrational modes are called \textbf{phonons}.
An electron can interact with a phonon, and this phonon can travel to another electron, interact, and thereby cause an indirect interaction between electrons.
The induced interaction between electrons is an example of second quantization.
They cause quantized interactions between electrons.

Phonons in solids can usually be described as harmonic oscillators.
\begin{equation}
    H = \frac{p^2}{2m} + \frac{K}{2} x^2 \label{1.3}
\end{equation}
Introducing the dimensionless coordinate $\xi$
\begin{eqnarray}
    \omega^2 &=& \frac{K}{m} \label{1.4}    \\
    \xi &=& x \sqrt{ \frac{m\omega}{\hbar} }     \label{1.5} \\
    -i \frac{\partial}{\partial \xi}  &=& \frac{p}{\sqrt{\hbar m \omega}} \label{1.6} \\
    H &=& \frac{\hbar \omega}{2} \left( - \frac{\partial^2}{\partial \xi^2} + \xi^2 \right) \label{1.7}
\end{eqnarray}
The harmonic oscillator Hamiltonian has a solution in therms of Hermit polynomials.
\begin{equation}
    H\psi_n = \hbar \omega(n+ \frac{1}{2} ) \psi_n   \label{1.8}
\end{equation}
Using the Dirac notation, the matrix elements for operators $x$ and $p$
\begin{eqnarray}
    \bra{n'}x\ket{n} &=& \sqrt{ \frac{\hbar}{2m\omega} } \left[\sqrt{n'}\delta_{n'}^{n+1} + \sqrt{n}\delta_{n'}^{n-1} \right] \nonumber \\
    \bra{n'}p\ket{n} &=& i\sqrt{ \frac{m\hbar\omega}{2} } \left[\sqrt{n'}\delta_{n'}^{n+1} - \sqrt{n}\delta_{n'}^{n-1} \right] \label{1.9}
\end{eqnarray}
For creation operators
\begin{eqnarray}
    a &=& \frac{1}{\sqrt{2}} \left( \xi + \frac{\partial}{\partial \xi} \right) = \sqrt{ \frac{m\omega}{2\hbar} } \left( x + \frac{ip}{m\omega} \right)   \nonumber \\
    a^\dagger &=& \frac{1}{\sqrt{2}} \left( \xi - \frac{\partial}{\partial \xi} \right) = \sqrt{ \frac{m \omega}{2 \hbar} } \left( x - \frac{ip}{m\omega} \right)   \label{1.10}
\end{eqnarray}
with commutation relations
\begin{equation}
    \left[ a, a^\dagger \right] = 1     \label{1.18}
\end{equation}
The eigenstate of \eqref{1.8} are
\begin{equation}
    \ket{n} = \frac{(a^\dagger)^n}{\sqrt{n!}} \ket{0} \label{1.22}
\end{equation}
and with
\begin{equation}
    a^\dagger\ket{n} = \sqrt{n+1} \ket{n+1} ~ ~ ~ a \ket{n} = \sqrt{n} \ket{n-1}    \label{1.28}
\end{equation}

The time dependence of these operator is, in Heisenberg representation of quantum mechanics
\begin{equation}
    O(t) = e^{iHt}Oe^{-iHt}~ ~ ~ \frac{\partial O(t)}{\partial t} = i \left[ H, O(t) \right] ~ \label{1.33}
\end{equation}
This suggest that
\begin{equation}
    a(t) = a \exp(-i\omega t) ~ ~ ~ a^\dagger(t) = a^\dagger \exp(i\omega t) \label{1.36}
\end{equation}
The time development of the position operator can be represented as
\begin{equation}
    x(t) = \sqrt{ \frac{\hbar}{2m\omega} } \left(a e^{-i\omega t} + a^\dagger e^{i\omega t} \right)     \label{1.37}
\end{equation}

Another problem is the charged harmonic oscillator in a constant electric field $F$
\begin{eqnarray}
    H &=& \frac{p^2}{2m} + \frac{K}{2} x^2 + eFx = \hbar \omega (a^\dagger a + \frac{1}{2} ) + \lambda(a+ a^\dagger)  \label{1.38} \\
    \lambda &=& eF \sqrt{ \frac{\hbar}{2m \omega} } \label{1.39}
\end{eqnarray}
This Hamiltonian may be solved exactly.
Consider the equation of motion for the time development
\begin{equation}
    \frac{\partial a}{\partial t} = i \left[ H, a\right] = - i \left( \omega a + \lambda \right)    \label{1.40}
\end{equation}
The right hand side is no longer just proportional to $a$, however, by define the new operators
\begin{eqnarray}
    A &=& a + \frac{\lambda}{\omega}  \label{1.41} \\
    A^\dagger &=& a^\dagger + \frac{\lambda}{ \omega}   \label{1.42} \\
    \frac{\partial A}{\partial t} &=& - i\omega A \label{1.43} \\
    A(t) &=& e^{-i \omega t} A    \label{1.44} \\
    A^\dagger (t) &=& e^{i\omega t} A^\dagger   \label{1.45}
\end{eqnarray}
This new operators have same properties of the Harmonic oscillator ones
\begin{equation}
    H = \omega \left( A^\dagger A + \frac{1}{2} \right) - \frac{\lambda^2}{\omega}  \label{1.50}
\end{equation}
with the eigenstate
\begin{eqnarray}
    H\ket{n} &=& \left[ \omega \left( n + \frac{1}{2} \right) - \frac{\lambda^2}{\omega} \right] \ket{n}    \label{1.51} \\
    \ket{n} &=& \frac{(A^\dagger)^n}{\sqrt{n!}} \ket{0} \label{1.52}
\end{eqnarray}
The position operator for this case is
\begin{equation}
    x(t) = \sqrt{ \frac{\hbar}{2m\omega} } \left( Ae^{-i\omega t} + A^\dagger e^{i\omega t} - \frac{2\lambda}{\omega} \right)   \label{1.53}
\end{equation}
The spring stretches to a new equilibrium point which is displaced a distance
\begin{equation}
    x_0 = \sqrt{ \frac{\hbar}{2m\omega} } \frac{2\lambda}{\omega} = - \frac{eF}{K} \label{1.54}
\end{equation}
from the original one.
This oscillator is in new equilibrium with the same frequency $\omega$ as before and still quantized in the unit of $\omega$.
The energy $ \frac{-\lambda^2}{\omega} = \frac{-e^2 F^2}{2K} $ is that gain by spring from the displacement along the electric field.
One can also get the same result by rewrite the Hamiltonian  as
\begin{equation}
    H = \frac{p^2}{2m} + \frac{K}{2} \left( x + \frac{eF}{K} \right)^2 - \frac{e^2F^2}{2K}  \label{1.55}
\end{equation}

In a solid system, there are many atoms, which mutually interact.
The vibrational modes are collective motions involving many atoms.
A simple introduction to this is by studying the normal modes of a one-dimensional harmonic chain
\begin{equation}
    H = \sum_i \frac{p_i^2}{2m } + \frac{K}{2} \sum_i \left( x_i - x_{i+1} \right)^2    \label{1.57}
\end{equation}
The classical solution is obtained by solving the equation of motion
\begin{equation}
    - m \ddot{x}_j = m \omega^2 x_j = K(2x_j -x_{j+1} - x_{j-1} )   \label{1.58}
\end{equation}
A solution is assumed of the form $x_j = x_0 \cos(k a j )$ and the force term becomes
\begin{equation}
    2 x_j - x_{j+1} - x_{j-1} = 2x_0 \cos(k a j) \left[ 1 - \cos(k a) \right]
\end{equation}
The normal modes have the solution
\begin{equation}
    \omega_k^2 = \frac{2K}{m} \left[ 1- \cos(k a) \right] = \frac{4K}{m} \sin^2( \frac{k a}{2} )    \label{1.59}
\end{equation}
The quantum mechanical solution begins by defining some normal coordinates, assuming periodic boundary condition
\begin{eqnarray}
    x_l &=& \frac{1}{\sqrt{N}} \sum_k e^{ikal} x_k; ~ ~ ~ x_k = \frac{1}{\sqrt{N}} \sum_l e^{-ikal} x_l \\
    p_l &=& \frac{1}{\sqrt{N}} \sum_k e^{-ikal} p_k; ~ ~ ~ p_k = \frac{1}{\sqrt{N}} \sum_l e^{ikal} p_l \label{1.60}
\end{eqnarray}
This choice maintains the desired commutation relations in either \textbf{real and wave vector} space
\begin{eqnarray}
    \left[x_l, p_m\right] &=& i \delta_{lm}~ ~ ~ \mathrm{real space} \label{1.61}   \\
    \left[x_k, p_p\right] &=& i \delta_{kp} \label{1.63}
\end{eqnarray}
It is easy to show that the Hamiltonian may written in wave vector space as\footnote{
    Since, $\sum_l x_l x_{l+m} = \sum_k x_k x_{-k} e^{iamk}$ and $\sum_k p_l^2 = \sum_k p_k p_{-k}$ and the definition \eqref{1.59}.
}
\begin{equation}
    H = \frac{1}{2m}  \sum_k \left[ p_k p_{-k} + m^2 \omega_k^2 x_k x_{-k} \right]  \label{1.64}
\end{equation}
The Hamiltonian has the form of a simple harmonic oscillator for each wave vector. Define the creation and destruction operators as
\begin{eqnarray}
    a_k &=& \sqrt{ \frac{m\omega_k}{2\hbar}}  \left( x_k + \frac{i}{m\omega_k} p_{-k} \right)  \label{1.65}  \\
    a^\dagger_k &=& \sqrt{ \frac{m\omega_k}{2\hbar} }  \left( x_k - \frac{i}{m\omega_k} p_{-k} \right)  \label{1.66}
\end{eqnarray}
Where they obey the commutation relations and have the same Hamiltion in the form of Harmonic oscillator.
These collective modes of vibration are called \textbf{phonons}.
They are the quantized version of the classical vibrational modes in the solid.
There are the same commutator relations and Hamiltonian as in the simple harmoic oscillator.
Each wave vector state behaves independently, with a possible set of quantum numbers $n_k=0,1,2,\dots$.
The state of the system at any time is
\begin{equation}
    \psi = \ket{n_1,n_2,\dots,n_n} = \Pi_k \ket{n_k} = \Pi_k \frac{(a^\dagger_k)^{n_k}}{\sqrt{n_k !}} \ket{0}   \label{1.71}
\end{equation}
with expectation value of Hamiltonian
\begin{equation}
    H= \sum_k \omega_k (n_k + \frac{1}{2})  \label{1.72}
\end{equation}
In thermal equilibrium the states have an averge value of $n_k$
\begin{equation}
    \langle n_k \rangle \equiv N_k = \frac{1}{e^{\beta \omega_k}-1} \label{1.73}
\end{equation}

The position operator in wave vector and real space is
\begin{eqnarray}
    x_k(t) &=& \sqrt{ \frac{\hbar}{2m\omega_k} } \left(a_k e^{-i\omega_k t} + a^\dagger_{-k} e^{i\omega_k t} \right) \nonumber \\
    x_l(t) &=&  \sqrt{ \frac{\hbar}{2mN\omega_k} }e^{ikal} \left(a_k e^{-i\omega_k t} + a^\dagger_{-k} e^{i\omega_k t} \right)    \label{1.74}
\end{eqnarray}
Often the term $mN$ is replaced by the quantity $mN = \rho V$, the mass density and volume.
Where the following notations are well know
\begin{eqnarray}
    \lim_{V\to \infty} \frac{1}{V} \sum_\bk f(\bk) &=& \int \frac{d^3 k}{8\pi^3} f(\bk)   \label{1.75} \\
    \lim_{V\to \infty} \delta_{\bk,\bk'} &=& \frac{8\pi^3}{V} \delta(\bk-\bk')    \label{1.78}
\end{eqnarray}

The quantum mechanical solution \eqref{1.64} has the same frequencies as found in the classical solution \eqref{1.58}.
Quantum mechanics only enters in a quantization of the amplitude of the oscillation.
The phonons occur in discrete numbers with zero, one, two, ..., phonons in state $\bk$.

In three-dimensional solids, the theroy is nearly identical except there are more indices.
Suppose there is a potential function between atoms or ions of the form
\begin{equation}
    \sum_{ij} V(R_i - R_j)  \label{1.79}
\end{equation}
where $R_i$ is the position of an atom.
Let $R_i^0$ as the equilibrium position and $Q_i$ as the displacement from equilibrium.
The potential function is expanded in a Taylor series about the equilibrum position
\begin{eqnarray}
    V(R_i-R_j) &=& V(R_i^0 - R_j^0) + (Q_i - Q_j) \nabla V(R_i^0 - R_j^0) \nonumber \\
    &+& \frac{1}{2} (Q_i - Q_j)_\mu (Q_i - Q_j)_\nu \frac{\partial^2}{\partial R_\mu \partial R_\nu}  V(R^0_i - R^0_j) \nonumber \\
    &+& O(Q^3)  \label{1.81}
\end{eqnarray}
The linear term in displacement vanishes, because of the definition of equilibrium point give the summation of forces is zero.
The important term is the one which is quadratic in the displacement.
It gives the potential eqnergy of the phonons
\begin{eqnarray}
    V_{ph} = \frac{1}{2} \sum_{ij} (Q_i-Q_j)_\mu (Q_i-Q_j)_\nu \Phi_{\mu\nu}(R_i^0 - R_j^0)     \label{1.83}
\end{eqnarray}
The interaction is evaluated in wave vector space by trying an expansion of the form
\begin{equation}
    Q_i(t) = i \sum_{\bk,\lambda} \sqrt{ \frac{\hbar}{2MN \omega_{\bk\lambda}} } \xi_{\bk,\lambda} \left( a_{\bk,\lambda} e^{-\omega_{\bk\lambda t}} + a^\dagger_{-\bk,\lambda} e^{i\omega_{\bk\lambda t}} \right) e^{i \bk \cdot R_i^0} \label{1.85}
\end{equation}
where $M$ is the ion mass.
The faction of $i$ on the right-hand side of the equation is required to make $Q_i^\dagger = Q_i$ since it represents a real displacement in space.
To make \eqref{1.85} id Hermitian it requires that
\begin{equation}
    \xi^*_{\bk,\lambda} = - \xi_{-\bk,\lambda}   \label{1.86}
\end{equation}
The polarization vector $\xi_{\bk,\lambda}$ are assumed to be real but change sign with $\bk$ direction, $\xi_{-\bk} = -\xi_\bk$ and the above identity is satisfied.
Since the displacement is in three diemnsions, there are $3L$ normal modes for each value of wave vector.
Where $L$ is the number of atoms per unit cell of the crystal.
THhe index $\lambda$ runs over these $3L$ values of normal mode.
Each mode has its own eigenfrequency.
It will also have a polarization vector $\xi_{\bk,\lambda}$ which specifies the bibrational direction of the ion for each wave vector and mode $\lambda$.
If there are more than one atom per unit cell, one should add further subscripts to $M$ and $\xi_{\bk,\lambda} $ to specify the values for each atom per unit cell.

The right-hand side of \eqref{1.81} may be written as
\begin{equation}
    = \sum_{ij} V(R^0_i - R^0_j) + \frac{M}{2} \sum_{\bk,\lambda} Q_{\bk,\lambda} Q_{-\bk,\lambda} \omega_{\bk\lambda}^2    \label{1.87}
\end{equation}
The first term is a constant which will be neglected in discussion of vibrational modes.
THe eigenvalues $\omega_{\bk\lambda}$ are thoes solved in the harmoic approximation.
In this approximation, one retains the quadratic term only in the displacements in the Hamiltonian.
To be more careful, a third order anharmonic term $V_3$ can be added for the case of one atom per unit cell\footnote{P10}.
The cubic terms $V_3$ permit on phonon to decay into two and vice versa.
