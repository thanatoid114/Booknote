\chapter{Exactly Solvable Models}

\section{Potential Scattering} \label{s4.1}
A very simple problem is an impurity potential $V(r)$ in an otherwise free-particle system.
All other interactions are ignored except that of the free particles with the impurity, which is assumed to be at the origin.
The potential is assumed to have no internal structure, and spherically symmetric.

The wave function for each particle may be obtained by solving a one-electron Schr{\"o}dinger equation
\begin{equation}
    H \psi_\lambda = \left[ - \frac{\hbar^2}{2m} \nabla^2 + V(r) \right] \psi_\lambda = \varepsilon_\lambda \psi_\lambda     \label{4.1}
\end{equation}

In many-body theory, the impurity problem is usually encountered as a scattering center.
The free-particle states are plane wave
\begin{equation}
    \psi(\br) = \frac{1}{\sqrt{v}} \sum_\bk C_{\bk\sigma} e^{i\bk \br}      \label{4.2}
\end{equation}
The Hamiltonian is expressed (without spin index) \footnote{Be careful about the notation of the Fourier transform \eqref{2.145}.}
\begin{eqnarray}
    H &=& \sum_\bk C_\bk^\dagger C_\bk + \frac{1}{v} \sum_{\bk\bk'} V_{\bk\bk'} C^\dagger_\bk C_{\bk'}  \label{4.3} \\
    V_{\bk\bk'} &=& \int d^3 r V(r) e^{-i\br(\bk-\bk')} = V(\bk-\bk')   \label{4.4}
\end{eqnarray}
The object is to diagonalize the Hamiltonian \eqref{4.3}, which the solutions are of course given by \eqref{4.1}.
However, the problem is not entirely solved.
Since the equation is second order, it has two solution.
These may be chosen as the ingoing and outgoing waves, or by combination give standing waves.
The choices must be related to the scattering problem implied in \eqref{4.3}.

The integral equation for the wave function is
\begin{eqnarray}
    \psi_\bk(\br) &=& \phi_\bk(\br) + \sum_{\bk'} \frac{\phi_{\bk'}(\br)}{\varepsilon_k - \varepsilon_{k'}} \int d^3r' \phi^*_{\bk'}(\br') V(r') \psi_\bk(\br') \label{4.5} \\
    \psi_\bk (\br) &=& \frac{1}{\sqrt{v}} e^{i \bk \br} \label{4.6} \\
\varepsilon_k &=& \frac{\hbar^2 k^2  }{2m}  \label{4.7}
\end{eqnarray}
This form of the integral equation is valid for the free-particle states with energy $\varepsilon_k$.
For bound states, the energy is changed to the binding energy $\varepsilon_k \to -\varepsilon_B$ where $\varepsilon_B >0$ and the term $\psi_\bk$ on the right is absent.

To prove \eqref{4.5} is equivalent to \eqref{4.1}. Operate on both sides of the equation by $H_0 - \varepsilon_k$,
\begin{equation}
    (H_0 -\varepsilon_k) \psi_\bk = (H_0 -\varepsilon_k) \phi_\bk + \sum_{\bk'} \frac{(H_0-\varepsilon_k) \phi_{\bk'}(\br)}{\varepsilon_k - \varepsilon_{k'}} \int d^3 r'\phi^*_{\bk'}(\br') V(r') \psi_\bk(\br)
\end{equation}
this gives
\begin{equation}
    (H_0 - \varepsilon_k) \psi_\bk = - \sum_{\bk'} \phi_{\bk'}(\br) \int d^3 r' \phi^*(\br') V(r') \psi_\bk(\br')   \label{4.9}
\end{equation}
The completeness relation for the summation over the set of states is
\begin{eqnarray}
    \sum_{\bk'} \phi_{\bk'} (\br) \phi_{\bk'}^* (\br') &=& \delta (\br- \br')   \label{4.10} \\
    (H_0 -\varepsilon_k) \psi_\bk &=& -V(r) \psi_\bk(\br) \label{4.11}
\end{eqnarray}
Then we have the desired answer
\begin{equation}
    (H_0 + V -\varepsilon_k) \psi_\bk (\br) = 0     \label{4.12}
\end{equation}

About the boundary conditions, the differential equation \eqref{4.1} is second order, so there are two independent solutions.
One choice is to have the wave function a standing wave.
This leads to the reaction matrix equation.
The other choices are the have the wave function an incoming wave or an outgoing wave, which leads to T-matrix theory.
The integral equation \eqref{4.5} is a convenient starting point for discussion of boundary conditions, since the various choices of standing, outgoing, or ingoing waves are determined only by the complex part of the energy denominator.
The factor $\varepsilon_k - \varepsilon_{k'}$, for standing waves, so that the principal part is chosen for the denominator.
The factor is $\varepsilon_k - \varepsilon_{k'} + i\delta$ for outgoing waves, and $\varepsilon_k -\varepsilon_{k'} - i \delta$ for ingoing waves.

\subsection{Reaction Matrix}
Here the energy denominator is chosen to be real and given by principal part.
The free-particle Green's function is then defined as
\begin{eqnarray}
    G_0(\bk,\br-\br') &=& P \sum_{\bk'} \frac{\phi_{\bk'}(\br) \phi^*_{\bk'}(\br')}{\varepsilon_k - \varepsilon_{k'}} = \frac{P}{v} \sum_{\bk'} \frac{e^{i\bk'(\br-\br')}}{\varepsilon_k - \varepsilon_{k'}}    \label{4.13} \\
    &=& P \int \frac{d^3 k'}{(2\pi)^3} \frac{e^{i\bk'(\br-\br')}}{\varepsilon_k - \varepsilon_{k'}}     \label{4.14}
\end{eqnarray}
The integral is standard and gives
\begin{equation}
    G_0(\bk,\br-\br') = \pi \rho(k) \frac{\cos \left[k \abs{\br-\br'} \right]}{k\abs{\br-\br'}}     \label{4.15}
\end{equation}
Where the density of stats of the particles
\begin{equation}
    \rho(k) = \int \frac{d^3 k'}{(2\pi)^3} \delta(\varepsilon_k - \varepsilon_{k'})   = \frac{mk}{2\hbar^2 \pi^2}   \label{4.16}
\end{equation}
The Green's function may be expanded as a function of $\br$ and $\br'$
\begin{equation}
    G_0(\bk,\br-\br') = \pi \rho(k) \sum_l (2l+1) P_l(\hat{\br}\hat{\br'}) j_l(k r_<) \eta_l(k_>)    \label{4.17}
\end{equation}
where the notation $r_<$ is the smaller of $\br$ and $\br'$.
The $j_l(kr)$ and $\eta_l(kr)$ are the spherical Bessel functions of the first and second kind, and $P_l(\cos \theta)$ are the Legendre functions.
\footnote{
\href{https://en.wikipedia.org/wiki/Bessel_function}{wiki:Bessel function} and
\href{https://en.wikipedia.org/wiki/Legendre_function}{wiki:Legendre function}
}
With \eqref{4.5}, we have
\begin{equation}
    \psi_\bk(\br) = \phi_\bk(\br) + \int d^3 r' G_0(\bk,\br-\br') V(r') \psi_\bk(\br')  \label{4.18}
\end{equation}
An important identity is the following, introducing the \textbf{angular momentum components $l$}, the plane wave may be expanded as
\begin{equation}
    e^{i\bk\cdot \br} = \sum_{l} \left( 2l+1 \right) i^l P_l(\hat{k}\cdot \hat{r}) j_l(kr)  \label{4.19}
\end{equation}
for the actual wave function
\begin{equation}
    \psi_\bk(\br) = \sum_l \left( 2l+1 \right) i^l P_l(\hat{k} \cdot \hat{r}) R_l(kr)   \label{4.20}
\end{equation}
The radial function $R_l(kr)$ satisfied the \textbf{radial Schr{\"o}dinger equation} of the form
\begin{equation}
    - \frac{\hbar^2}{2m} \left[ \frac{1}{r^2} \frac{\partial}{\partial r} r^2 \frac{\partial R}{\partial r} \frac{l(l+1)}{r^2} R \right] + \left[ V(r) - \varepsilon_k \right] R = 0    \label{4.21}
\end{equation}
which does not determine the boundary conditions.
Using the fact that,
\begin{equation}
    \int d \Omega_r P_l(\hat{k}\cdot\hat{r}) P_m(\hat{r}\cdot\hat{p}) = \frac{4\pi}{2l+1} \delta_{lm} P_l(\hat{k}\cdot\hat{p})  \label{4.22}
\end{equation}
to reduce the equation down to one which involves only the same angular momentum component,
\begin{eqnarray}
    R_l(kr) &=& j_l(kr) + 4\pi^2 \rho(k)  \int_0^\infty r'^2 dr' j_l(kr_<) \eta_l(kr_>) V(r') R_l(kr') \nonumber \\
    &=& j_l(kr) + 4\pi^2 \rho(k) \left[ \eta_l(kr) \int_0^r r'^2 dr' j_l(kr') V(r') R_l(kr') \right. \nonumber \\
    &~ ~ &  + \left. j_l(kr) \int_r^\infty r'^2 dr' \eta_l(kr') V(r') R_l(kr') \right] \label{4.23}
\end{eqnarray}
It is important that the potential is spherically symmetric.
Otherwise the scattering term would mix angular momentum components, which make it hard to solve.

The solution is examined in the limit as $kr\to \infty$.
From \eqref{4.21} it can be shown that the radial wave function must asymptotically approach the value
\begin{equation}
    \lim_{kr\to \infty} R_l(kr) \to \frac{C_l(k)}{kr} \sin \left[kr + \delta_l(k) - \frac{l\pi}{2}  \right] \label{4.24}
\end{equation}
The prefactor $C_l$ is determined below. The asymptotic limit of the integral equation \eqref{4.23} is
\begin{equation}
    \lim R_l(kr) \to j_l(kr) + D_l(k) \eta_l(kr) \to \frac{\sin(kr-l\pi/2)}{kr} - D_l \frac{\cos(kr- l\pi/2)}{kr}   \label{4.25}
\end{equation}
where
\begin{equation}
    D_l(k) = 4\pi^2 \rho(k) \int_0^\infty r^2 dr j_l(kr) V(r) R_l(kr)   \label{4.26}
\end{equation}
The potential $V(r)$ is assumed to be of short range: \textbf{It falls off faster than $r^{-2}$ at large distance}.
The scattering is then described by a phase shift $\delta_l(k)$ which depends on angular momentum and wave vector.
The two asymptotic expansions \eqref{4.24} and \eqref{4.25} must be identical, which means
\begin{equation}
    D_l = - \tan (\delta_l) = 4\pi^2 \rho(k) \int_0^\infty r^2 dr j_l(kr) V(r) R_l(kr)  \label{4.27}
\end{equation}
The complicated integral just defines the tangent of the phase shift. Then \eqref{4.25} becomes
\begin{equation}
    \lim_{kr\to\infty} R_l(kr) \to \frac{1}{kr\cos(\delta)} \left[\cos(\delta) \sin(kr-l\pi/2) + \sin(\delta) \cos(kr-l\pi/2) \right] = \frac{\sin(kr + \delta_l l\pi/2)}{kr \cos(\delta_l)} \label{4.28}
\end{equation}
The normalization coefficient in \eqref{4.24} is
\begin{equation}
    C_l = \frac{1}{\cos(\delta_l)}  \label{4.29}
\end{equation}
When solving the radial wave function \eqref{4.21}, the solution is obtained which is will behaved at the origin.
This solution is followed outward in $r$ until the region is reached where $V(r) \approx 0$ and the centrifugal barrier $\hbar^2 l (l+1)/(2mr^2)$ is small.
Then the solution has the form \eqref{4.24} with $C_l = 1/\cos(\delta_l)$.
The wave function is now properly normalized at large $r$, and by following it back toward the origin it is normalized everywhere.
These steps provide the proper solution to the scattering equation \eqref{4.5} which was the starting point in the calculation.

The reaction matrix is defined as the quantity
\begin{equation}
    R_{\bk' \bk} = \int d^3 \phi^*_{\bk'}(\br) V(r) \psi_\bk(\br)  \label{4.30}
\end{equation}
Expanding in angular momentum state by using the expansions for the wave function and the plane wave, the angular integrals give
\begin{equation}
    R_{\bk'\bk} = 4\pi \sum_l(2l+1) P_l(\hat{k}\cdot \hat{k'} ) R_l(k',k)   \label{4.31}
\end{equation}
where the radial integral is
\begin{equation}
    R_l(k',k) = \int_0^\infty r^2 dr j_l(k'r) V(r) R_l(kr)  \label{4.32}
\end{equation}
It is defined for the general case where $k' \neq k$.
And if they are equal, the answer is just \eqref{4.27}
\begin{equation}
    R_l(k,k) = -\frac{\hbar^2 \tan(\delta_l)}{2mk}  \label{4.33}
\end{equation}

The reaction matrix obeys an integral equation which is deduced by putting \eqref{4.30} into the integral equation \eqref{4.5} for $\psi_\bk(\br)$
\begin{equation}
    R_{\bk'\bk} = V_{\bk'\bk} + P \sum_{\bk_1} \frac{V_{\bk'\bk_1} R_{\bk_1 \bk}}{\varepsilon_k -\varepsilon_{k_1}} \label{4.34}
\end{equation}
