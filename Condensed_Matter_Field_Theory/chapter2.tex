\chapter{Second quatization}
Second quntization provides a compact way of representing the many-body space of excitations;
secondly, the properties of the ladder operatore, $\hat{a}_k$ were encoded in a simple set of commutation relations rather than in some explicit Hilbert space representation

\section{Introduction to second quantization}\label{sec:2.1}
\subsection{Motivation}
Consider the normalized set of wavefunctions $\ket{\lambda}$ of some single-particle Hamiltonian $\hat{H}: \hat{H}\ket{\lambda} = \epsilon_\lambda \ket{\lambda}$.
With this definition, two-particle wavefunction in two leverls is
\begin{equation}
\psi_{F,B}(x_1,x_2) = \frac{1}{\sqrt{2}} \left(\braket{x_1}{\lambda_1} \braket{x_2}{\lambda_2} \mp \braket{x_1}{\lambda_2} \braket{x_2}{\lambda_1} \right)
\end{equation}
More generally, an appropriately symmetrized $N$-particle wavefunction can be expressed in \textbf{Slater determinants}
\begin{equation}
    \ket{\lambda_1, \lambda_2,\dots, \lambda_N} \equiv \frac{1}{\sqrt{ N ! \prod^\infty_{\lambda=0}\left( n_\lambda ! \right)}} \sum_\mathcal{P} \zeta^{(1 - \mathrm{sgn} \mathcal{P})/2}  \ket{\lambda_{\mathcal{P}_1}} \otimes \ket{\lambda_{\mathcal{P}_2}} \otimes \dots \otimes \ket{\lambda_{\mathcal{P}_N}}
    \label{eq:2.1}
\end{equation}
where $\eta_\lambda$ represents the total number of particles in state $\lambda$.
The summation runs over $N!$ permutations of the set of quantume numbers ${\lambda_1,\dots \lambda_N}$ and $\mathrm{sgn}\mathcal{P}$ denotes the sign of the permutation $\mathcal{P}$ \footnote{ Permutation form $\left(\mathcal{P}_1,\dots \mathcal{P}_N \right) \to \left( 1,\dots, N\right)$.}.

While representations eq.~\eqref{eq:2.1} can be used to represent the full Hilbert space of many-body quanatume mechanics, it is not always convenient:
\begin{itemize}
    \item Eq.~\eqref{eq:2.1} is cumbersome.
    \item For the problem with fixed number of particle
    \item A represention where the quantum numbers of individual quasi-particles rather than the entangled ste of quantum number of all consitituents are fundamental.
\end{itemize}

\subsection{The apparatus of second qunatization}
\newthought{Occupation number representation}, in this representation, the basis states of $F^N$ are specified by $\ket{n_1,n_2,\dots}$.
Any state $\ket{\Psi}$ in $F^N$ can be obtain by a linear superposition.

Define the \textbf{Fock space} as
\begin{equation}
    \mathcal{F} \equiv \oplus_{N=0}^\infty \mathcal{F}^N
    \label{eq:2.2}
\end{equation}
The complicated permutation "entanglement" implied in the definition \eqref{eq:2.1} of the Fock state can be generated by application of a set of linear operators to a single reference state.
\begin{equation}
    \ket{n_1, n_2, \dots} = \prod_i \frac{1}{\sqrt{n_i !}} \left( a_i^\dagger \right)^{n_i} \ket{0}
    \label{eq:2.4}
\end{equation}

\marginnote{
    Suppose that $\mathcal{A}$ is irreducibly represented in some vector space $V$, i.e. that there is a mapping assigning to each $a_i \in \mathcal{A}$ a linear mapping $a_i: V\to V$, such that every vecotr $\ket{v} \in V$ can be reached from and other $\ket{w} \in V$ by application fo r operatore $a_i$ and $a_i^\dagger$.
    According to the \textbf{Stone-von Neumann theorem} (a) such a representation is unique upt to unitary equivalence; (b) there is a unique state $\ket{0} \in V$ that is annihilated by every $a_i$.
}

For practical aspects we need to find out, for Fock space, how changes from one single-particle basis to another affect the operator algebra, and in what way generic operators acting in many-particle Hilbert spaces can be represented in terms of creation and annihilation operators.

\textbf{Change basis}
\begin{equation}
    a^\dagger_{\lambda'} = \sum_\lambda \braket{\lambda}{\lambda'} a^\dagger_\lambda,~ ~  a_{\lambda'} = \sum_\lambda \braket{\lambda'}{\lambda} a_\lambda
    \label{eq:2.8}
\end{equation}

\textbf{Representation of operators}: Single-particle operators acting in the N-partilce Hilbert space generally take the form $\hat{\mathcal{O}}_1 = \sum_{n=1}^N \hat{o}_n$, where $\hat{o}_n$ is an ordinary single-particle operator acting on the $n$th particle.
In general, from a representation to a general basis,
\begin{equation}
    \hat{\mathcal{O}}_1 = \sum_{\mu\nu} \bra{\mu} \hat{o} \ket{\nu} a^\dagger_\mu a_\nu
    \label{eq:2.11}
\end{equation}
for two-bdy operator in general we have
\begin{equation}
    \hat{\mathcal{O}}_2 = \sum_{\lambda\lambda'\mu\mu'} \bra{\mu,\mu'} \hat{\mathcal{O}}_2 \ket{\lambda,\lambda'} a^\dagger_\mu a^\dagger_{\mu'} a_\lambda a_{\lambda'}
    \label{eq:2.16}
\end{equation}

\section{Application of second qunatization}\label{sec:2.2}
