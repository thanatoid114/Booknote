\chapter{GROUP REPRESENTATIONS}

\section{Representations}
\textbf{Definition 2.1} (Representations of a Group): If there is a homomorphism from a group $G$ to a group operatures $U\left(G\right)$ on a linear vector space $V$, we say that $U\left(G\right)$ forms a \textit{representation} of the group $G$.
The \textit{dimension of the representations} is the dimension of the vector space $V$.
A representation is said to be \textit{faithful} if the homomorphism is alo an isomorphism.
A \textit{degenerate representation} is one which is not faithful.

To be more specific: the representation is a mapping
\begin{equation*}
  g \in G \xRightarrow[above]{U} \left(g\right)
\end{equation*}
where $U\left(g\right)$ is an operator on $V$, such that
\begin{equation}
  \label{eq:2.1-1}
  U\left(g_{1}\right) U\left(g_{2}\right) = U \left(g_{1} g_{2}\right)
\end{equation}
i.e., the representation operators satisfy the same rules of multiplication as the original group elements.

Consider the case of a finite-dimensional representation.
Choose a set of basis vectors on $V$.
The operators $U\left(g\right)$ are then realized as $n\times n$ matrices $D\left(g\right)$ as follows\footnote{where $j$ is the row index and $i$ is the column index}:
\begin{equation}
  \label{eq:2.1-2}
 U\left(g\right) \ket{e_{i}} = \ket{e_{j}} D\left(g\right)^{j}_{i}, ~ ~ ~ g \in G.
\end{equation}
And from Eq.~\eqref{eq:2.1-1}, we have
\begin{equation}
  \label{eq:2.1-3}
 D\left(g_{1}\right) D\left(g_{2}\right) = D\left(g_{1}g_{2}\right)
\end{equation}
And since $D\left(G\right)$ satisfy the same algebra as $U\left(G\right)$, the group of matrices $D\left(G\right)$ forms a \textit{matrix representation} of $G$.

\textrm{Example 1}: The trivial $1-$dimenional representation for every group $G$.
Let $V=C$, and $U\left(g\right) = 1$ for all $g \in G$.
Then $g\in G \longrightarrow 1$ forms a representation.

\textrm{Example 2}: Let $G$ be a group of matrices, $V=C$, and $U\left(g\right) =\det g$.
This define non-trivial one-dimenional representation since $\det g_{1} \det g_{2} = \det g_{1}g_{2}$.

\textrm{Example 4}: Let $G$ be the dihedral group $D_{2}$ consisting of $e$ (identity), $h$ (reflection about the $Y-$axis), $v$ (reflection about the $X-$axis), and $r$ (rotation by $\pi$ around the origin) as described at Tab.~\ref{tab:1-3} and Fig.~\ref{fig:1-1}.
Let $V_{2}$ be the two-dimensional Euclidean space with basis vector $\left(\hat{e}_{1}, \hat{e}_{2}\right)$.
By referring to Fig.~\ref{fig:2-1} and making use of the definition Eq.~\eqref{eq:2.1-2}, we easily infer
\begin{equation}
  \label{eq:2.1-4}
  D\left(e\right) =
  \begin{bmatrix}
    1 & 0 \\
    0 & 1
  \end{bmatrix}
  D\left(h\right) =
  \begin{bmatrix}
    -1 & 0 \\
    0 & 1
  \end{bmatrix}
  D\left(v\right) =
  \begin{bmatrix}
    1 & 0 \\
    0 & -1
  \end{bmatrix}
  D\left(r\right) =
  \begin{bmatrix}
    -1 & 0 \\
    0 & -1
  \end{bmatrix}
\end{equation}
It is straightforward to very that the mapping $g\longrightarrow D\left(g\right)$ is a homomorphism, hence these matrices form a $2-$dimensional representation of the $D_{2}$ group.
\begin{figure}
  \centering
  \begin{tikzpicture}
    \draw [<->] (-2,0) -- (2,0);
    \draw [<->] (0,-2) -- (0,2);
    \draw [dashed] (0,0) circle [radius=2];
    \node [right] at (2,0) {$\hat{e}_{1}, ~ v\hat{e}_{1}$};
    \node [above] at (0,2) {$\hat{e}_{2}, ~ h\hat{e}_{2}$};
    \node [left] at (-2,0) {$r\hat{e}_{1}, ~ h\hat{e}_{1}$};
    \node [below] at (0,-2) {$v\hat{e}_{2}, ~ r\hat{e}_{2}$};
  \end{tikzpicture}
  \caption{$D_2$ transformations}
  \label{fig:2-1}
\end{figure}

\textrm{Example 5}: Let $G$ be the group of continuous rotations in a plane around origin $O$, $G=\{R\left(\phi\right), 0\leq \phi \leq 2\pi\}$.
Let $V_{2}$ be the two-dimensional Euclidean space, since
\begin{align}
  \label{eq:2.1-5}
  \hat{e}_{1}' &= \hat{e}_{1} \cos \phi + \hat{e}_{2} \sin \phi \\
  \hat{e}_{2}' &= -\hat{e}_{1} \sin \phi + \hat{e}_{2} \cos \phi \nonumber
\end{align}
we have
\begin{equation}
  \label{eq:2.1-6}
  D\left(\phi\right) =
  \begin{bmatrix}
    D_{1}^{1}=\cos \phi & D_{1}^{2}=-\sin \phi \\
    D_{2}^{1}=\sin \phi & D_{2}^{2}=\cos \phi
  \end{bmatrix}
\end{equation}
Since according Eq.~\eqref{eq:2.1-2}, $U \ket{e_{1}} = \ket{e_{1}} D\left(\phi\right)_{1}^{1}+ \ket{e_{2}} D\left(\phi\right)_{1}^{2}$.
However, if $\mathbf{x}$ is an arbitrary vector in $V_{2}$, $\mathbf{x} = \hat{e}_{j} x^{j}$, then
\begin{align}
  \label{eq:2.1-7}
  \mathbf{y} &= U\left(\phi\right) \mathbf{x} = \hat{e}_{j} y^{j} \\ \nonumber
  y^{j} &= D\left(\phi\right)^{j}_{i} x^{i} \\ \nonumber
  \begin{bmatrix}
    y^{1} \\
    y^{2}
  \end{bmatrix}
  &=
  \begin{bmatrix}
  \cos \phi & -\sin \phi \\
  \sin \phi & \cos \phi
  \end{bmatrix}
  \begin{bmatrix}
  x^{1}\\
  x^{2}
  \end{bmatrix}
\end{align}

As we can see, different groups may be realized on the same vector space.
Here, the $2-$dimensional Euclidean space $V_{2}$ is seen to provide representations for the two finite groups $D_{2}$, $D_{3}$ as well as the continuous group $R\left(2\right)$

\textrm{Example 7}: Let $V_{f}$ be te space of complex-valued linear homogeneous founctions $f$ of two real varibales $\left(x,y\right)$:
\begin{equation}
  \label{eq:2.1-8}
  f\left(x,y\right) = a x + by
\end{equation}
where $\left(a,b\right)$ are arbitary complex coefficients.
Interpret $\left(x,y\right)$ as the components of a vector $\mathbf{x}$ in a $2$-dimenional Euclidean space $V_{2}$.
Then the group operations from previos examples will induce the following transformation in the function space $V_{f}$
\begin{equation}
\label{eq:2.1-9}
f \xRightarrow{g \in G} f'\left(x^{1},x^{2}\right) \equiv f \left(x'^{1},x'^{2}\right)
\end{equation}
where $\mathbf{x}' = U\left(g^{-1}\right) \mathbf{x}$.
It is straightforward to show that the mapping defined by Eq.~\eqref{eq:2.1-9} is a homomorphism; for if $g" g' = g$ then
\begin{align}
\label{eq:2.1-10}
f \xRightarrow{g'} f' ~ ~ ~ f' \left(\mathbf{x}\right) &= f \left[ U\left(g'\right)^{-1} \mathbf{x} \right] \\ \nonumber
f' \xRightarrow{g"} f" ~ ~ ~ f"\left( \mathbf{x} \right) &= f' \left[ U\left(g"\right)^{-1} \right] = f \left[U\left(g'\right)^{-1} U\left(g"\right)^{-1} \mathbf{x} \right] \\ \nonumber
  &= f\left[U\left(g"g'\right)^{-1} \mathbf{x}\right] = f\left[ U\left(g\right)^{-1} \mathbf{x}\right]
\end{align}
Therefore, the set of transformation defined by Eq.~\eqref{eq:2.1-9} forms a representation of the group $G$.

\textbf{Theorem 2.1:} (i) If the group $G$ has a non-trivial invariant subgroup $H$, then any representation of the factor group $K=G/H$ is also a representaion of $G$.
This representaion must be degenerate; (ii) Conversely, if $U \left( G \right)$ is a degenerate representation of $G$, then $G$ has at least one invariant subgroup $H$ such that $U \left( G \right)$ defines a faithful representation of the factor group $G/H$.

A immediate corollay of this theorem is that all representations of simple groups are faithful.

\section{Irreducible, Inequivalent Representations}
The first type of redundacy is due to similarity transformation.

\textbf{Definition 2.2} (Equivalence of Representations): Two representations of a group $G$ related by a similarity transfromation are said to be equivalent.

By seeking characterizations of the representation which are invariant under similarity transfomations, we can tell the two representations are equivalant or not.
The trace $\tr A$ and determinant $\det A$ are two characterizations which are independent of the choice of the base.

\textbf{Definition 2.3} (Characters of a Representation): The \textit{character} $\chi \left( g \right)$ of $g \in G$ in a represenation $U \left( G \right)$ is defined to be $\chi \left( g \right) = \Tr U \left( g \right)$.
All group elements in a given class of $G$ have the same characters, becase $\Tr D \left( p \right) D \left( g \right) D \left( p^{-1} \right)= \Tr D \left( g \right)$.
Therefor, the group character is a function of the class-label only.

A second type of redundacy concerns direct sum representations.

\textbf{Definition 2.4} (Invariant Subspace): Let $U \left( G \right)$ be a representation of $G$ on the vector space $V$, and $V_{1}$ be a subspace of $V$ with the property that $U \left( g \right) \ket{x} \in V_{1}$ for all $\mathbf{x} \in V_{1}$ and $g \in G$.
$V_{1}$ is siad to be an \textit{invariant subspace} of $V$ with repsect ot $U \left( G \right)$.
An invariant subspace is \textit{mimimal} or \textit{proper} if it does not contain any non-trival invariant subspace with respect ot $U \left( G \right)$.

Examples of trivial invariant subspaces of $V$ with respect to $U \left( G \right)$ are: (i) the space $V$ itself, and (ii) the subspace consisting only of the null vector.

\textbf{Definition 2.5} (Irreducibel Representations): A representation $U \left( G \right)$ on $V$ is \textit{irreducible} if there is no non-trivial invariant subspace in $V$ with respect to $U \left( G \right)$.
Otherwise, the representation is \textit{reducible}.
In latter case, if the orthogonal complement\footnote{If $V_{1}$ is a subspace of $V$, the orthogonal complement of $V_{1}$ consists of all vectors in $V$ which are orthogonal to very vector in $V_{1}$. For finite-dimensional vector spaces, at least, the orthogonal complement of $V_{1}$ also forms a subspace, called $V_{2}$, then we have $V = V_1 \oplus V_2$.}of the invariant subspace is also invariant with respect to $U \left( G \right)$, then the represenation is said to be \textit{fully reducible or decomposable.}

\textrm{Example 1}: Consider the action of the dihedral group $D_2$ on the $2-$dimensional Euclidean space $V_2$ as described in Fig.~\ref{fig:2-1} and Eq.~\eqref{eq:2.1-4}.
The $1-$dimensional subspace spanned by $\hat{e}_1$ is invariant under all four group operations, $D_2 \left( g \right) \hat{e}_1 = \pm \hat{e}_1$.
The same is true for the subspace spaned by $\hat{e}_2$.
Thus the $2-$dimensional representation of the group given by Eq.~\eqref{eq:2.1-4} is therefore a reducible representation.

\textrm{Example 2}: The $1-$dimensional subspace spanned by $\hat{e}_{1}$ or $\hat{e}_2$ is not invariant under the group $\mathbf{R} \left( 2 \right)$.
However, if we form the following linear combinations of vectors,
\begin{equation}
  \label{eq:2.2-1}
 \hat{e}_{\pm} = \frac{1}{\sqrt{2}} \left( \mp \hat{e}_1 - i \hat{e}_2 \right)
\end{equation}
it is straightforward to show that:
\begin{align}
  \label{eq:2.2-2}
  U \left( \phi \right) \hat{e}_+ &= \hat{e}_+ e^{-i\phi} \\ \nonumber
  U \left( \phi \right) \hat{e}_- &= \hat{e}_- e^{i\phi}
\end{align}
Therefore, the $1-$dimensional spaces spanned by $\hat{e}_{\pm}$ are individually invariant under the rotation group $R \left( 2 \right)$.
The $2-$dimensional representation given by Eq.~\eqref{eq:2.1-6} can be simplified if we make a change of basis to the eigenvectors $\hat{e}_{\pm}$.
In this new basis,
\begin{equation}
  \label{eq:2.2-3}
  D' \left( \phi \right) =
  \begin{bmatrix}
    e^{-i\phi} & 0 \\
    0 & e^{i\phi}
  \end{bmatrix}
\end{equation}
The new matrices can be obtained form the old one in Eq.~\eqref{eq:2.1-6} by a similarity transformation which is defined by Eq.~\eqref{eq:2.2-1}.

Let us look at the general matrix form of a reducible representation.
If $V_1$ is an $n_1-$dimensional invariant subspace with respect to $U \left( G \right)$, we can always choose a set of basis vectors in $V$ such that the first $n_1$ vectors are in $V_1$.
Since, for all $g\in G$
\begin{equation}
 U \left( g \right) \ket{e_i} = \ket{e_j} D \left( g \right)^j_i \in V_1 ~ ~ ~ \text{for}~ i= 1, \dots, n_{1} \nonumber
\end{equation}
we conclude that $D \left( g \right)_i^j = 0$ for $i=1, \dots, n_1$ and $j= n_1+1, \dots, n$.
Therefor, the matrix representation is of the form
\begin{equation}
  \label{eq:2.2-4}
  D \left( g \right) =
  \begin{bmatrix}
    D_1 \left( g \right) & D' \left( g \right) \\
    0 & D_2 \left( g \right)
  \end{bmatrix}
\end{equation}
If $D \left( g \right)$ and $D \left( g' \right)$ are both of this form, then the product $D \left( gg' \right)$ is also of this form, and that $D_i \left( gg' \right) = D_i \left( g \right) D_i \left( g' \right)$ for $i=1,2$.
Thus, all esential properties of $D \left( G \right)$ are already contained in the representations $D_i \left( G \right)$.

We see that if $U \left( G \right)$ is a representation of the group $G$ on $V$ and $V^{\mu}$ is an invariant subspace of $V$ respect to $G$, then by restricing the action of $U \left( G \right)$ to $V^{\mu}$, we obtain a lower-dimension representation $U^{\mu} \left( G \right)$.
If the subspace $V^{\mu}$ cannot be further reduced, $U^{\mu} \left( G \right)$ is an irreducible represenation, and we say that $V^{\mu}$ is a \textrm{proper or irreducible invariant subspace} with respect to $G$.

\section{Unitary Representations}
\textbf{Definition 2.6} (Unitary Representation): If the group representation space is a inner product space, and if the operators $U \left( g \right)$ are unitary for $g \in G$, then the representation $U \left( G \right)$ is said to be a \textit{unitary representation}.

Because symmetry transformation are naturally associated with unitary operators, unitary representations play a cnetral role in studying symmetry groups.

\textbf{Theorem 2.2:} If a unitary representation is reducible, then it is alos decomposable i.e., fully reducible.

\textbf{Theorem 2.3:} Every representation $D \left( G \right)$ of a finite group on an inner product space is equivalent to a unitary representation, i.e., there exist a similar transformation such that $S D \left( g \right) S^{-1}$ is unitary.






%%% Local Variables:
%%% mode: latex
%%% TeX-master: "main"
%%% End:
